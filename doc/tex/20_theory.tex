Im folgenden Abschnitt gibt es eine kurze Einführung zum Thema \enquote{\gls{machine-learning}}. % CHECK emph statt enquote
Danach werden die für diese Arbeit relevanten Problemstellungen formuliert,
bevor die Datenquellen beschrieben werden.
Nachdem die Datenquellen bekannt sind,
wird die Erstellung von Trainingsdaten beschrieben,
woraufhin kurz auf die Architektur von \gls{BERT} eingegangen wird.


\subsection{\Gls{machine-learning}}


\subsubsection{Bewertung \glslink{machine-learning}{maschinellen Lernens}}
Die folgenden Metriken werden oft genutzt,
um die Güte eines Modells darzustellen:
\gls{nlp:stats:loss},
\gls{nlp:stats:recall},
\gls{nlp:stats:precision} und
\gls{nlp:stats:f1}.

Zum besseren Verständnis werden im folgenden einige Begriffe eingeführt,
bevor diese Metriken definiert werden.
\begin{defn}[Typische Benennung bei binärer Kategorisierung]
	Data ist hierbei die \enquote{Wahrheit},
	während \enquote{Prediction} die Kategorisierung durch das System ist.
	\begin{center}
		\begin{tabularx}{0.8\textwidth}{l l | c c | r}
			                               &                   & \multicolumn{2}{c|}{\textbf{Prediction}} &                                      \\
			                               &                   & \textbf{Positive}                        & \textbf{Negative}  & $\sum$          \\
			\cline{1-5}
			\multirow{2}{*}{\textbf{Data}} & \textbf{Positive} & True Positive                            & False Negative     & Actual Positive \\
			                               & \textbf{Negative} & False Positive                           & True Negative      & Actual Negative \\
			\cline{1-5}
			                               & $\sum$            & Predicted Positive                       & Predicted Negative &
		\end{tabularx}
	\end{center}
	\captionof{table}{\gls{confusion-matrix}, welche die typischen Bezeichnungen der Merkmalsausprägungen für eine binäre Klassifikation zeigt}
\end{defn}
\FloatBarrier

Für die Nachvollziehbarkeit wird in \cref{tbl:exampledistribution} die \gls{confusion-matrix} mit beispielhaften Zahlen gefüllt.
% CHECK non floating
\begin{table}[H]
	\begin{center}
		\begin{tabularx}{0.5\textwidth}{l l | c c | r}
			                               &                   & \multicolumn{2}{c|}{\textbf{Prediction}} &                            \\
			                               &                   & \textbf{Positive}                        & \textbf{Negative} & $\sum$ \\
			\cline{1-5}
			\multirow{2}{*}{\textbf{Data}} & \textbf{Positive} & 5                                        & 5                 & 10     \\
			                               & \textbf{Negative} & 90                                       & 9900              & 9990   \\
			\cline{1-5}
			                               & $\sum$            & 95                                       & 9905              &
		\end{tabularx}
	\end{center}
	\caption{Beispielhafte Verteilung}
	\label{tbl:exampledistribution}
\end{table}

Die Formeln für diese sind nach \citeauthor{towardsdatascience:stats}
wie folgt \autocite{towardsdatascience:stats} definiert:

\begin{defn}[\glspt{nlp:stats:precision}]
	\begin{equation}
		\label{eqn:precision}
		\text{Precision}
		= \frac{\text{True Positive}}{\text{True Positive}+\text{False Positive}}
		= \frac{\text{True Positive}}{\text{Predicted Positive}}
	\end{equation}
\end{defn}

Somit wäre für \cref{tbl:exampledistribution}
die Precision
\(\frac{5}{95} \approx 5,3\% \).

\begin{defn}[\glspt{nlp:stats:recall}]
	\begin{equation}
		\label{eqn:recall}
		\text{Recall}
		= \frac{\text{True Positive}}{\text{True Positive}+\text{False Negative}}
		= \frac{\text{True Positive}}{\text{Actual Positive}}
	\end{equation}
\end{defn}

Für unser Beispiel \cref{tbl:exampledistribution}
ergibt sich ein Recall von \(\frac{5}{10} = 50\% \).

\begin{defn}[\glspt{nlp:stats:f1}]
	\begin{equation}
		\label{eqn:f1}
		\text{F1}
		= 2 \times \frac{\text{Precision}\times\text{Recall}}{\text{Precision}+\text{Recall}}
	\end{equation}
\end{defn}

Der F1-Wert ist also
\(2 \times \frac{\frac{5}{95}\times\frac{5}{10}}{\frac{5}{95}+\frac{5}{10}}=\frac{2}{21} \approx 9,5\%\).


\nocite{overleaf:HowToThesisPart3}
\begin{table}
	\begin{subtable}{0.45\textwidth}
		\begin{center}
			\begin{tabularx}{\textwidth}{l l | c c | r}
				                               &                   & \multicolumn{2}{c|}{\textbf{Prediction}} &                            \\
				                               &                   & \textbf{Positive}                        & \textbf{Negative} & $\sum$ \\
				\cline{1-5}
				\multirow{2}{*}{\textbf{Data}} & \textbf{Positive} & 10                                       & 0                 & 10     \\
				                               & \textbf{Negative} & 9990                                     & 0                 & 9990   \\
				\cline{1-5}
				                               & $\sum$            & 10 000                                   & 0                 &
			\end{tabularx}
		\end{center}

		\caption{Optimistisches Modell}
		\label{tbl:optimisticmodell}
	\end{subtable}
	\hfill
	\begin{subtable}{0.45\textwidth}
		\begin{center}
			\begin{tabularx}{\textwidth}{l l | c c | r}
				                               &                   & \multicolumn{2}{c|}{\textbf{Prediction}} &                            \\
				                               &                   & \textbf{Positive}                        & \textbf{Negative} & $\sum$ \\
				\cline{1-5}
				\multirow{2}{*}{\textbf{Data}} & \textbf{Positive} & 0                                        & 10                & 10     \\
				                               & \textbf{Negative} & 0                                        & 9990              & 9990   \\
				\cline{1-5}
				                               & $\sum$            & 0                                        & 10 000            &
			\end{tabularx}
		\end{center}
		\caption{Pessimistisches Modell}
		\label{tbl:pessimisticmodell}
	\end{subtable}
	\caption{Konstante Modelle}
	\label{tbl:constant-modells}
\end{table}

Sei ein Modell gegeben, 
welches immer positiv bzw. immer negativ vorhersagt,
dann sehen die Matritzen wie in \cref{tbl:optimisticmodell} bzw. in \cref{tbl:pessimisticmodell} aus.
Diese erhalten die Werte in \cref{tbl:examplemetricsoverview}.

\begin{table}
	\begin{center}
		\begin{tabularx}{0.5\textwidth}{l r r r}
			          & Optimistic & Pessimistic & Example   \\
			Precision & \(0,1\%\)  & NA          & \(5,3\%\) \\
			Recall    & \(100\%\)  & \(0\%\)     & \(50\%\)  \\
			F1        & \( 100\%\) & NA          & \(9.5\%\)
		\end{tabularx}
	\end{center}
	\caption{Tabellarischer Vergleich der Metriken an einem optimistischen, einem pessimistischen und dem ursprünglichen-Beispiel Modell}
	\label{tbl:examplemetricsoverview}
\end{table}

\begin{defn}[\glspt{nlp:stats:loss}\autocite{1906.01378}]
	Seien
	\(\mathbf{X} \in \mathcal{X}\)
	und
	\(Y \in \mathcal{Y}\)
	Zufallsvariablen,
	wobei
	\(\mathcal{X}\subset\mathbb{R}^d\)
	und
	\(\mathcal{Y} = \left\lbrace0,1\right\rbrace\).
	Eine \gls{nlp:stats:loss}[-Funktion]
	ist eine Abbildung
	\(\glssymbol{nlp:stats:loss}: \mathbb{R} \times \mathcal{Y} \to \mathbb{R}^{+}.\)
	Hierbei ist ein kleinerer Wert besser.
	\foreigntextquote{english}[\autocite{google:trainingandloss}]{Loss is the penalty for a bad prediction}
\end{defn}

Bei \gls{nlp:stats:loss} wird also nicht ein gutes Modell \enquote{belohnt},
sondern ein schlechtes Modell \enquote{bestraft}.
Für eine solche \gls{nlp:stats:loss}[ Funktion] \glssymbol{nlp:stats:loss}
lässt sich das \glssymbol{nlp:stats:loss}[-Risiko]
wie folgt definieren:

\begin{defn}[{\glssymbol{nlp:stats:loss}[-Risiko]\autocite{1906.01378}}]
	Seien
	\(\mathbf{X} \in \mathcal{X}\)
	und
	\(Y \in \mathcal{Y}\)
	Zufallsvariablen,
	wobei
	\(\mathcal{X}\subset\mathbb{R}^d\)
	und
	\(\mathcal{Y} = \left\lbrace0,1\right\rbrace\).
	Für eine beliebige \gls{nlp:stats:loss}[-Funktion] \glssymbol{nlp:stats:loss}
	und eine Klassifikationsfunktion \(f\),
	wird das \glssymbol{nlp:stats:loss}[-Risiko] definiert als:
	\begin{equation}
		\label{eqn:loss-risk}
		R_{\glssymbol{nlp:stats:loss}} \left(f\right)
		=
		\mathbb{E}_{\mathbf{X},Y} \glssymbol{nlp:stats:loss}\left(f(x) \mid y_{x}\right)
	\end{equation}
	wobei \(\mathbb{E}\) den Erwartungswert bezeichnet
	und der Index die Zufallsvariablen bezüglich derer er gebildet wird.
\end{defn}


\subsubsection{Arten \glslink{machine-learning}{maschinellen Lernens}}
\Gls{machine-learning} ist eine Unterkategorie von \gls{artificialintelligence}.
Das Ziel ist dem Computer die Möglichkeit zu geben zu lernen,
ohne explizit dafür programmiert zu sein \autocite{levity:howdomachineslearn}.

\citeauthor{levity:howdomachineslearn} unterteilt es in die drei folgenden großen Kategorien:


\begin{defn}[\gls{supervised-learning} \autocite{levity:howdomachineslearn}]
	\Gls{machine-learning}
	auf Basis von bereits annotierten Tupeln der Form
	\(\left(X, f(X)\right)\),
	wobei \(X\) eine Eingabe und \(f(X)\) die erwartete Ausgabe ist,
	nennt man \gls{supervised-learning}.
	Solche Tupel werden oft auch als \texttt{labeled data} bezeichnet.
\end{defn}

Hierbei sind nach \citeauthor{levity:howdomachineslearn}
die Hauptaufgaben Regression und Klassifikation.

Bei Klassifikation ist die Funktion definiert als
\(f: \mathcal{X} \to \mathcal{Y}  \),
wobei \(\mathcal{Y} = \left\lbrace \text{\texttt{LABEL}}_i \mid i=1 \vdots l \right\rbrace \)
die Menge der möglichen Klassen ist.
Der Wertebereich ist also diskret.\autocite{ledu:regression-versus-classification}
Bei Regression hingegen ist der Wertebereich kontinuierlich
und es gilt im Allgemeinen \(f: \mathcal{X} \to \mathbb{C}^{l}\)
bzw.\, \(f: \mathcal{X} \to \mathbb{R}^{l}\) .

Das so trainierte Modell ist ein Funktionsapproximation für \(f\).
Es gibt für eine Regressionsaufgabe eine Fortsetzung und Vereinfachung der bisherigen Funktion
und für Klassifikationsaufgaben Label für bisher unbekannte Eingaben.

Neuere Methoden reduzieren den Bedarf an Trainingsdaten,
indem sie diese nach bestimmten Regeln erzeugen.
Dies kann durch Einbindung externer Systeme geschehen,
wie z.B.\, bei \gls{distant-supervision} und \gls{reinforcement-learning}
oder nur auf Grundlage der Daten.

\begin{defn}[\gls{distant-supervision}\autocite{deepdive:stanford:distant-supervision}]
	Unter Verwendung einer bestehenden \gls{knowledgebase}
	wird die Eingabe mit einer erwarteten Ausgabe kombiniert,
	um \texttt{labeled data} für \gls{supervised-learning} zu erzeugen.
\end{defn}

\begin{defn}[\gls{reinforcement-learning} \autocite{levity:howdomachineslearn}]
	Hierbei gibt im Gegensatz zu \gls{supervised-learning} keine Tupel,
	sondern eine bewertende Instanz,
	die mitteilt,
	ob eine bestimmte Aktion oder Entscheidung des Modells
	\enquote{gut} oder \enquote{schlecht} war.
\end{defn}

\begin{defn}[\gls{unsupervised-learning} \autocite{levity:howdomachineslearn}]
	Beim \gls{unsupervised-learning} sind die Daten nicht weiter klassifiziert.
	Die Tupel \(\left(X, f(X)\right)\) können also nur automatisch erzeugt werden.
	Diese Art des Lernens nennt man auch \textquote[\autocite{ng:deeplearning}]{self-taught learning}.
\end{defn}

\label{ssec:dartstellbarkeit}
Da die Modelle meist durch Tensorberechnungen abgebildet werden,
ist eine Darstellung der Informationen als dünnbesetzter Vektor
von Vorteil.
So müssen weniger einzelne Berechnungen durchgeführt werden
und gleichzeitig sinkt durch eine solche Darstellung der Speicherbedarf.\autocite{10.1007/bf02331346}
Die Darstellbarkeit von Informationen als dünnbesetzte Vektoren 
in einer entsprechenden Basis wurde von \citeauthor{olshausen1996emergence}
bereits \citeyear{olshausen1996emergence} in \citetitle{olshausen1996emergence}
gezeigt. \autocite{olshausen1996emergence}
Solch eine Darstellung kann ein System
durch Optimierung nach einer \gls{nlp:stats:loss}[-Funktion],
die Informationserhaltung und Dünnbesetztheit betrachtet,
im Rahmen des \gls{unsupervised-learning}
einfach lernen. \autocite[Formel 2-4]{olshausen1996emergence}
% CHECK Add examples here


\subsection{\glsfmtfull{naturallanguageprocessing}}
\input{tex/22_nlp.tex}

\subsection{Datenquellen}
Im Folgenden werden die zwei Datenquellen,
die für diese Arbeit verwendet wurden,
beschrieben.
Die erste Datenquelle ist die \gls{bllontology},
die für die Erstellung der Annotationen benötigt wird.
Danach folgt das \glsfmtfull{oai-pmh},
welches die zu annotierenden Daten liefert.

\subsubsection{\glspt{bllontology}}


\begin{defn}[Thesaurus\autocite{Arano:ThesaurusAndOntologies}] 
	Ein Thesaurus 
	ist eine Art von dokumentierender
	Sprache,
	die die konzeptuelle Struktur des Wissens eines bestimmten Fachgebietes
	darstellt.
	Er stellt den semantischen Zusammenhang zwischen verschieden Konzepten
	durch ein eingeschränktes Vokabular für die Beziehungen
	dar.
\end{defn}

Das Ziel eines Thesaurus ist \enquote{synonymy},
also die Überlappung der Bedeutung von Wörtern \autocite{oxfordbibliographies:Synonymy},
und \enquote{polysemy},
also die Mehrdeutigkeit der Bedeutung innerhalb eines Bereiches \autocite{oxfordbibliographies:Polysemy},
zu reduzieren.

Im Folgenden werden Beispiele von Dateien in der \gls{turtle}[ Syntax] gezeigt \autocite{w3c:turtle}.
Diese wird daher kurz erklärt.

Die Syntax besteht im Grunde nur aus Tripeln von \emph{Subjekt}, \emph{Prädikat} und \emph{Objekt}.
Zur Vereinfachung bleibt das aktuelle \emph{Subjekt} bis zum nächsten \emph{Punkt (.)}
und das aktuelle \emph{Prädikat} bis zum nächsten \emph{Semikolon (;)} aktiv.
Ein \emph{Komma (,)} trennt in diesem Zusammenhang die \emph{Objekte}.
Bei StringLiterals 
ist eine \gls{i18n}
möglich.
\mintinline{turtle}{"TEXT"@de} ist somit als String für die Sprache Deutsch definiert.

Damit die Konzepte eineindeutig sind,
werden sie über \gls{uri} beschrieben;
zur Vereinfachung können diese aber durch global erklärte Präfixe ersetzt werden.
Ein Beispiel hierfür ist in \Cref{lst:bll:thesaurus:prefixes}.

\begin{listing}
	\begin{tcolorbox}[tlistingstyle]
		\inputminted[
			firstline=1,
			lastline=13
		]{turtle}{../data/bll-thesaurus.ttl}
	\end{tcolorbox}
	\caption{Erklärung der globalen Präfixe in der \texttt{bll-thesaurus.ttl}}
	\label{lst:bll:thesaurus:prefixes}
\end{listing}

Nach dem die Präfixe
erklärt sind,
sind die zwei beispielhaften Einträge in \Cref{lst:bll:thesaurus:seneca} eindeutig definiert.
Der Eintrag \mintinline{turtle}{bllt:bll-133103862}
ist eine Klasse und ein Konzept.
Er wird durch \mintinline{turtle}{"Seneca"}
sowohl auf Deutsch,
als auch auf Englisch
und sowohl für das Konzept,
wie auch für die Anzeige klassifiziert.
Er ist eine Unterklasse von \mintinline{turtle}{bllt:BLLConcept}.
Als gröbere Klassifizierung gilt \mintinline{turtle}{bllt:bll-133108791} (Irokesisch).
Die Notation \mintinline{turtle}{"02.25.01.047.06"} repräsentiert diese Hierarchie,
wie in \Cref{fig:bll:notation:seneca}
zu sehen ist.

\begin{listing}[ht]
	\begin{tcolorbox}[tlistingstyle]
		\inputminted[
			firstline=12171,
			lastline=12183
		]{turtle}{../data/bll-thesaurus.ttl}
	\end{tcolorbox}
	\caption{Beispieleinträge aus \texttt{bll-thesaurus.ttl}}
	\label{lst:bll:thesaurus:seneca}
\end{listing}

\begin{figure}[ht]
	% https://latexdraw.com/draw-trees-in-tikz/
	\nocite{latexdraw:trees}
	\begin{tcolorbox}[tfigurestyle]
		\begin{center}
			\begin{tikzpicture}
	[
		level1/.style = {fill=red},
		level2/.style = {fill=orange},
		level3/.style = {fill=yellow},
		level4/.style = {fill=green},
		level5/.style = {fill=cyan},
		level6/.style = {fill=violet!70!white},
		every node/.append style = {draw, anchor = west},
		grow via three points={one child at (0.5,-0.8) and two children at (0.5,-0.8) and (0.5,-1.6)},
		edge from parent path={(\tikzparentnode\tikzparentanchor) |- (\tikzchildnode\tikzchildanchor)}]
	 
	
	\node[level1] {00}
		child {node[level1] {BLL-Klassifikation} edge from parent [dashed]}
		child {
			node[level2] {02}
			child {node[level2] {Nicht-indoeuropäische Sprachen} edge from parent [dashed]}
			child {
				node[level3] {02.25}
				child {node[level3] {Indigene amerikanische Sprachen} edge from parent [dashed]}
				child {
					node[level4] {02.25.01}
					child {node[level4] {Indigene Sprachen Nordamerikas und Zentralamerikas} edge from parent [dashed]}
					child {
						node[level5] {02.25.01.047}
						child {node[level5] {Irokesisch} edge from parent [dashed]}
						child {
							node[level6] {02.25.01.047.06 }
							child {node[level6] {Seneca} edge from parent [dashed]}
						}
					}
				}
			}
		};
	\end{tikzpicture}
		\end{center}
	\end{tcolorbox}
	\caption{Darstellung der Notation des Eintrages \Cref{lst:bll:thesaurus:seneca} als Baum mit den übergeordneten Klassen}%
	\label{fig:bll:notation:seneca}
\end{figure}

\FloatBarrier
\begin{defn}[\glspt{ontology}\autocite{10.1016/S0169-023X:97:00056-6}\autocite{Arano:ThesaurusAndOntologies}]
	\foreignblockquote{english}[{\autocite[Abschnitt 1.]{10.1006/knac.1993.1008}  und \autocite[Abschnitt 2.1]{Borst1997} via \autocite[Abschnitt 6.1]{10.1016/S0169-023X:97:00056-6}} ]{An ontology is a formal, explicit specification of a shared conceptualisation}
	wobei die folgenden Definitionen gelten:

	\begin{enumerate}
		\item Konzeptualisierung
		      : Ein abstraktes Modell eines Phänomens der realen Welt basierend auf den relevanten Konzepten des Phänomens
		\item Explizit: der Typ des Konzepts und die Einschränkungen seiner Nutzing sind explizit definiert
		\item Formal: Die Syntax ist präzise genug, um von einem Computer verstanden zu werden
		\item Geteilt: das Wissen ist von einer Gruppe akzeptiert
	\end{enumerate}

	Die Konzeptualisierung
	wird von \citeauthor{Arano:ThesaurusAndOntologies}
	noch weiter spezifiert,
	sodass sie eine Perspektive eine bestimmten Realität involvieren muss
	und diese auf der konzeptuellen Struktur einer \gls{knowledgebase}
	begründet wird.

	Das Ziel einer \gls{ontology}
	ist das Teilen des Wissens,
	welches sie repräsentiert.
\end{defn}
\FloatBarrier

Da sich die Präfixe in der \gls{ontology}
von denen im Thesaurus 
unterscheiden,
werden sie in \Cref{lst:bll:ontoloy:prefixes} noch einmal explizit dargestellt.

\begin{listing}
	\begin{tcolorbox}[tlistingstyle]
		\inputminted[
			firstline=1,
			lastline=6
		]{turtle}{listings/turtle:shortened:bll-ontology.ttl}
	\end{tcolorbox}
	\caption{Erklärung der globalen Präfixe in der \texttt{bll-ontology.ttl}}
	\label{lst:bll:ontoloy:prefixes}
\end{listing}

Nachdem die Präfixe
erklärt sind,
sind die zwei beispielhaften Einträge in \Cref{lst:bll:ontology:seneca} eindeutig definiert.
Das erste Beispiel ist wieder \mintinline{turtle}{"Seneca"}.
Dieses Mal ist die Klassifizierung als \enquote{\mintinline{turtle}{a skos:Concept}} % CHECK linebreak
nicht mehr enthalten,
dafür ist die neue Klassifikation 
als \enquote{\mintinline{turtle}{rdfs:subClassOf bllt:NorthernIroquoian}} % CHECK format
hinzugekommenen.
Das Format dieser \enquote{Elternklasse} zeigt an,
dass es eine manuell angelegte Klasse ist,
da der Name innerhalb des \texttt{bllt}-Namespaces keine Nummer ist.
Das zweite Beispiel zeigt zusätzliche Prädikate in der \gls{owl}[-Syntax],
wie Versionierungsinformationen und Äquivalenz.

\begin{listing}
	\begin{tcolorbox}[tlistingstyle]
		\inputminted[
			firstline=8,
			lastline=25
		]{turtle}{listings/turtle:shortened:bll-ontology.ttl}
	\end{tcolorbox}
	\caption{Leicht umformatierte Beispieleinträge aus \texttt{bll-ontology.ttl}
		(Der Zeilenumbruch nach dem Subjekt ist jeweils entfernt,
		damit das Syntaxhighlighting mit \gls{pygments} funktioniert)
	}
	\label{lst:bll:ontology:seneca}
\end{listing}


\begin{defn}[\glsfmtfull{bll}]
	Die \gls{bll} ist eine Bibliographie,
	also ein Verzeichnis von Literaturnachweisen
	zu linguistischer Literatur,
	insbesondere der Allgemeinen Linguistik
	und \textquote[\autocite{linguistik:de:kataloge:info}]{der anglistischen, germanistischen und romanistischen Sprachwissenschaft}.
\end{defn}

Zu dem oben genannten Verzeichnis existiert der \gls{bll}[ Thesaurus],
welcher für die Indizierung im Portal verwendet wird.
Bei der Umwandelung des \gls{bll}[ Thesaurus] in die \gls{bllontology},
die initial von \citeauthor{L16-1707} durchgeführt wird \autocite{L16-1707},
wird der Umfang an Begriffen eingeschränkt
auf Konzepte aus den Zweigen \foreigntextquote{english}[\autocite{data:linguistic:ontology-doc}]{Syntax, Morphology, Lexicology and Phonology}
und einigen zusätzlichen Einträgen aus den Zweigen \foreigntextquote{english}[\autocite{data:linguistic:ontology-doc}]{Graphemics and Semantics}.
Daher ist z.B.\,  der Term \texttt{bllt:bll-467296421} aus \Cref{lst:bll:thesaurus:seneca}
nicht in der \gls{bllontology} verzeichnet.

\subsubsection{\glsfmtfull{oai-pmh}}
Die \glsfmtfull{openarchivesinitiative} hat mit \gls{oai-pmh}
einen Standard für Interoperabilität zwischen verschiedenen Metadaten-Quellen
und Dienstanbietern definiert.
In \Cref{lst:oaipmh:xml:begin} ist der Anfang einer mit \gls{oaipmharvest} erzeugten
\gls{xml}[-Datei] dargestellt.

\begin{longlisting}
	\begin{tcolorbox}[tlistingstyle]
		\inputminted[
			firstline=1,
			lastline=21,
			fontsize=\footnotesize,
			escapeinside=||,
			gobble=10
		]{xml}{listings/oaixml:shortened:2022-07-13__oai_dc__000000000000.xml}
	\end{tcolorbox}
	\nopagebreak
	\caption{Beispiel Beginn eines \gls{oai-pmh}[-Eintrags]}
	\label{lst:oaipmh:xml:begin}
\end{longlisting}

Da der Inhalt innerhalb des \mintinline{xml}{oai_dc:dc}-Tags
durch das \gls{dublin-core}\footnote{\url{https://www.dublincore.org/specifications/dublin-core/dces/1999-07-02/}}
festgelegt ist,
liegt hier der Fokus.
Durch den Standard sind
der Titel (\mintinline{xml}{dc:title}),
das Thema (\mintinline{xml}{dc:subject}), welches oft als Liste von Schlagworten ausgedrückt wird,
Beschreibungen (\mintinline{xml}{dc:description})
und
eindeutige Referenzen (\mintinline{xml}{dc:identifier}) auf die Ressource
als Einträge definiert.
Auch die verwandten Ressourcen (\mintinline{xml}{dc:relation})
und das Format (\mintinline{xml}{dc:format})
sind enthalten.

Wenn der Volltext benötigt wird,
müssen die \mintinline{xml}{dc:identifier} oder die \mintinline{xml}{dc:relation}-Einträge
betrachtet werden.
Je nach Anbieter können hier bereits Referenzen auf Volltext-Dokumente hinterlegt sein
oder die Referenzen weisen nur auf Webseiten,
auf welchen die Referenzen auf die Volltext-Dokumente gefunden werden müssen.


Sowohl für \Cref{prob:nlp:ner} als auch für \Cref{prob:nlp:nel}
werden annotierte Daten benötigt.

\subsection{Erstellung von Trainingsdaten}
Da Trainingsdaten annotierte Daten sind,
wird nachfolgend auf die Arten und die Erstellung von Annotationen eingegangen.

\subsubsection{Arten der Annotation}

Eine mögliche Annotation ist es,
den Wörtern eines Satzes ihre grammatikalische Funktion
oder andere semantische Informationen
zuzuordnen.
Wenn die semantische Information ist,
dass die annotierte Entität eine \enquote{spezielle} Entität ist,
welche einen Namen hat,
dann spricht man von \gls{namedentityrecognition}.
Auch eine Annotation über die Referenz auf die Entität ist möglich.

Da es keine bestehenden Annotationen für die Daten gibt,
sollen hier Möglichkeiten für die automatische Erzeugung betrachtet werden.

\subsubsection{Automatische Erzeugung von Annotationen}

\citeauthor{2006.15509} listet als Möglichkeiten für die automatische Annotation
\enquote{Stringmatching}, \enquote{\glssymbol{regex}} und heuristische Verfahren
\autocite{2006.15509}.

Bei ersterem wird ein Match nur erkannt,
wenn es exakt ist.
Dies führt dazu,
dass die Wahrscheinlichkeit Wörter falsch zu erkennen,
sehr niedrig ist.
Die Rate der False Negative entsprechend steigt.
Des Weiteren werden ausschließlich vorher bekannte Entitäten erkannt.
So wird \enquote{Müller und Sohn} erkannt,
wenn es vorher aufgelistet wird.
Der Transfer,
dass \enquote{Meier und Sohn} ebenfalls eine Entität ist,
kann nicht stattfinden.
Dazu kommt das Problem,
dass insbesondere in Grammatiken,
in denen Flexion
wichtig ist,
die gleiche Entität mit verschiedenen Strings repräsentiert werden kann.
Die Laufzeit ist relativ hoch und nicht parallelisierbar.
So liegt die Laufzeit von \Cref{alg:stringmatching:longestmatch}
in \(\mathcal{O}\left( n \times k + l_E \right)\),
wobei \(l_E\) die Anzahl der Wörter in der Wortliste beschreibt.
Zusätzlich wird für das \enquote{Stringmatching} eine Wortliste benötigt.
Eine Liste von Entitäten kann
z.B.\, durch Extraktion aus einer \gls{knowledgebase}
erstellt werden.
In \autocite[A.1]{2006.15509} von \citetitle{2006.15509}
wird von \citeauthor{2006.15509} eine ganze Reihe von solchen \glspl{knowledgebase}
genannt.
Eine solche Liste von Entitäten wird oft als \gls{gazetteer} bezeichnet
(siehe z.B. \autocite[Introduction]{Carlson2009}).

Zur Erhöhung der Trefferrate
kann die Liste erweitert werden.
In \autocite[Abschnitt 3.4]{OASIcs-LDK-2019-11}
werden Ansätze zur Erweiterung der Liste erklärt.
Die Erkennungsaufgabe wird in der eben genannten Arbeit
von \gls{finitestatetransducer}
durchgeführt,
welche eine Erkennung zuerst auf der Wortebene durchführen.
Im Beispiel von oben würden so die Worte
\enquote{Müller}, \enquote{und} und \enquote{Sohn} auf der Zeichenebene erkannt werden
und danach die Entität \enquote{Müller und Sohn} auf der Wortebene.

Der Begriff Wortebene passt nur eingeschränkt,
denn in Sprachen,
die wie das Deutsche von zusammengesetzen Wörtern leben,
ist eine Einteilung in Wortbestandteile
(diese werden in \autocite{OASIcs-LDK-2019-11} als \foreignquote{english}{lemma} bezeichnet)
zielführender.

Der Ansatz mit der Wortebene zeigt auf einen konzeptuell-anderen Ansatz:
Es gibt bestimmte Formulierungen oder \enquote{Muster},
die oftmals eine Entität markieren.
So ist ein \enquote{NAME und \{Sohn,Söhne\}} insbesondere,
wenn es in Anführungszeichen steht,
oftmals eine Firma.
Dementsprechend wird den Token dieser Sequenz
die Kategorie \gls{nlp:category:org} zugeordnet.
Auch Ketten von Großbuchstaben sind oftmals Entitäten (Acronyme).
Beide Beispiele lassen sich mit \glssymbol{regex} darstellen,
siehe \cref{lst:regex:and-sons,lst:regex:acronym}.

\begin{listing}
	\begin{minipage}{\linewidth}
		\begin{tcolorbox}[tlistingstyle]
			\mintinline{perl}{(?P<NAME>\w*)\s+und\s+(Sohn|Söhne)}
		\end{tcolorbox}
		\subcaption[a]{\enquote{Name und Sohn} bzw.\, enquote{Name und Söhne}}% HACK
		\label{lst:regex:and-sons}
	\end{minipage}
	\begin{minipage}{\linewidth}
		\begin{tcolorbox}[tlistingstyle]
			\mintinline{perl}{(?P<acronym>[[:<:]][[:upper:]]+[[:>:]])\s+(?P<description>\((.|\s)+\))?}
		\end{tcolorbox}
		\subcaption[a]{Acronyme}% HACK
		\label{lst:regex:acronym}
	\end{minipage}
	\caption{Beispielhafte \glslink{regex}{reguläre Ausdrücke}}
\end{listing} % CHECK linebreak

Die verschiedenen Ausgaben,
der oben genannten \gls{finitestatetransducer},
lassen sich als eine \glssymbol{regex} pro Ausgabe ansehen.
Um die Ausführung zu beschleunigen,
ist also eine parallele Prüfung mit verschiedenen \glssymbol{regex} denkbar.
Das Problem,
welches dabei auftritt,
ist die Ambiguität,
da den gleichen Token verschiedene Annotationen zugewiesen werden können.

Mit heuristischen Ansätzen,
wie sie z.B.\, in \autocite{1906.01378}
beschrieben werden,
kann eine Entscheidung getroffen werden,
welches Annotation die \enquote{beste} ist.
So könnte die häufigste oder auch die längste ausgewählt werden.

In dieser Arbeit wird aus Zeitgründen ausschließlich \enquote{Stringmatching} implementiert.

\subsubsection{Stringmatching}
\label{ssec:stringmatching}

Stringmatching wird,
wie in \citetitle{1906.01378} beschrieben,
auch hier für die Generierung der Trainingsdaten verwendet.

\begin{algorithm}
	\begin{tcolorbox}[talgostyle]
		\begin{algorithmic}
			\Require Wortliste $D_E$, Satz $s = \left\lbrace w_1, \hdots, w_m \right\rbrace$, Kontextlänge $k$
			%
			\State $i \gets 1$
			\State $l_1 \hdots l_m \gets \texttt{unlabeled} $
			\While{$i < m$}
			\For{$j=k$; $j--$}
			\If{$w_i \hdots w_{i+j} \in D_E$}
			\State $l_i, \hdots, l_{i+j}  \gets \texttt{Positive} $
			\State $i \gets i+j+1$
			\State BREAK
			\EndIf
			\If{$j==0$}
			\State $i \gets i + 1$
			\State BREAK
			\EndIf
			\EndFor
			\EndWhile
			\Return teilweise annotierter Satz $\left(
				\left\lbrace
				w_1, \hdots, w_m
				\right\rbrace,
				\left\lbrace
				l_1, \hdots, l_m
				\right\rbrace
				\right)$
			%
		\end{algorithmic}
	\end{tcolorbox}
	\caption{Label mit Stringmatching: Longest Match nach \autocite{1906.01378}}%
	\label{alg:stringmatching:longestmatch}
\end{algorithm}

Der ursprüngliche~\Cref{alg:stringmatching:longestmatch}
wird in \Cref{alg:stringmatching:optimized}
wie folgt optimiert:
Die verbleibende Länge des zu betrachtenden Strings wird in Betracht gezogen,
um keine Matches zu suchen,
die über den betrachteten String hinausragen.
Zusätzlich werden nur Längen versucht,
denen ein Wort in der Wortliste entspricht.
Als Eingabe wird zusätzlich der Typ $t$,
welcher für die Token genutzt werden soll,
und ein Label $l$ verwendet,
sodass der gleiche Satz mit mehreren Wortlisten nacheinander annotiert werden kann.
Hierbei werden bereits bestehende Label nicht überschrieben.

\begin{algorithm}
	\begin{tcolorbox}[talgostyle]
		\begin{algorithmic}
			\Require Wortliste $D_E$, Satz $s = \left\lbrace w_1, \hdots, w_m \right\rbrace$,Label  $l = \left\lbrace l_1, \hdots, l_m \right\rbrace$, Kontextlänge $k$, Token $t$
			\State $lengths \gets $ Liste der Wortlängen von $D_E$ von gross nach klein sortiert
			\State $lenghts \gets lengths + 0$
			\While{$i < m$}
			\For{$j \in lengths$ if $j < (m-1)$}
			\If{ $w_i, \hdots, w_{i+j} \in D_E$}
			\If{$l_v == \texttt{unlabeled} \forall v=i,\hdots,i+j$}
			\State $l_i \gets "B-"+t$
			\State $l_{i+1}, \hdots, l_{i+j}  \gets "I-"+t $
			\State $i \gets i+j+1$
			\State BREAK
			\EndIf
			\If{$j==0$}
			\State $i \gets i + 1$
			\State BREAK
			\EndIf
			\EndIf
			\EndFor
			\EndWhile
			\Return teilweise annotierter Satz $\left(
				\left\lbrace
				w_1, \hdots, w_m
				\right\rbrace,
				\left\lbrace
				l_1, \hdots, l_m
				\right\rbrace
				\right)$
		\end{algorithmic}
	\end{tcolorbox}
	\caption{Label mit Stringmatching: Longest Match with Optimisations}\label{alg:stringmatching:optimized}
\end{algorithm}

Um das Named Entity Linking zu vereinfachen,
wird für jeden \gls{gazetteer}
eine eigene Klasse angelegt.
So muss nach den jeweiligen Entitäten nur in einer \gls{knowledgebase} gesucht werden.
Für die Einheitlichkeit wird der Name des \gls{gazetteer} in Großbuchstaben verwendet.
Z.B.\, \texttt{BLL-DE} für Einträge der \gls{bllontology} auf Deutsch.
Inspiration hierfür sind die detaillierten \glspl{gazetteer}
von \citeauthor{Leitner2019} in \citetitle{Leitner2019} \autocite{Leitner2019}.

Dieser Algorithms wurde zur Vereinfachung von \gls{namedentitylinking}
um die Eingabe einer Liste von Entitätsreferenzen erweitert
und die Wortliste $D_E$ durch ein Wörterbuch ersetzt,
sodass den Namen eine eindeutige Referenz auf die Entität zuordnet.

\subsection{\gls{BERT}}
\glsfmtfull{BERT} ist ein Modell für \gls{naturallanguageprocessing}.
Es wurde von \citeauthor{1810.04805} auf Basis von \gls{transformer} entworfen,
Diese Architektur beruht vor allem auf Aufmerksamkeitsmechanismen,
welche in \autocite{1706.03762} vorgestellt wurde.
Der Vorteil dieser Architektur bei \gls{naturallanguageprocessing}
gegenüber den vorherigen \gls{lstm}-basierenden Architekturen ist es,
dass beidseitiger Kontext für die Bewertung der Eingaben herangezogen werden kann.

\subsubsection{Training des Modells}
Das Modell wird auf \texttt{unlabeled data}
mit zwei verschiedenen Aufgaben trainiert.
Beim \gls{nlp:task:mlm}
werden randomisiert 15\% der Token maskiert.
Da in späteren Aufgaben keine \texttt{[MASK]}
Token mehr auftauchen,
werden 10\% der Token durch ein anderes Token
und 10\% der Token durch sich selbst \enquote{maskiert}.
\autocite[3.1 Task \# 1]{1810.04805}.
Dieser Task ermöglicht dem Modell
eine kontextuelles \enquote{Verständnis}
einzelner Wörter bzw. Token im Satz zu erlernen.

Die zweite Aufgabe ist \gls{nlp:task:nsp}.
Hierbei werden zwei Sätze A und B aus dem Corpus gewählt,
sodass in 50\% der Fälle Satz B auf Satz A folgt
und in 50\% der Fälle B ein zufälliger Satz aus dem Corpus ist.
\autocite[3.1 Task \# 2]{1810.04805}.
Die kontextuelle Einbettung des  \texttt{[CLS]} Tokens
wird hierbei für die Auswertung betrachtet.
Das Modell lernt somit Zusammenhänge zwischen Sätzen.

Insgesamt entwickelt das Modell eine Darstellung der Informationen,
wie sie in~\Cref{ssec:dartstellbarkeit}
beschrieben wird.
Diese Darstellung wird von den \texttt{heads} verwendet,
um die eigentlichen Aufgabenstellungen zu lösen.
Verschiedenen \texttt{heads} sind am Beispiel der \gls{huggingface:transformers}[-Bibliothek]
in \Cref{tbl:transformers-heads} aufgelistet.

\begin{table}
	\begin{tabularx}{\linewidth}{lccp{3cm}p{3cm}}
		\toprule
		\multicolumn{5}{c}{\textbf{Heads}}                                                                                                                                   \\
		Name                    & Input                         & Output                   & Tasks                                       & Ex. Datasets                      \\
		\midrule
		Language Modeling       & $x_{1:n-1}$                   & $x_n \in {\cal V}$       & Generation                                  & WikiText-103                      \\
		Sequence Classification & $x_{1:N}$                     & $y \in {\cal C}$         & Classification, \newline Sentiment Analysis & GLUE, SST, \newline MNLI          \\
		Question Answering      & $x_{1:M},$ \newline $x_{M:N}$ & $y$ span $[1:N]$         & QA,  Reading\newline Comprehension          & SQuAD, \newline Natural Questions \\
		Token Classification    & $x_{1:N}$                     & $y_{1:N} \in {\cal C}^N$ & NER, Tagging                                & OntoNotes, WNUT                   \\
		Multiple Choice         & $x_{1:N}, {\cal X}$           & $y \in {\cal X}$         & Text Selection                              & SWAG, ARC                         \\
		Masked LM               & $x_{1:N\setminus n}$          & $x_n \in {\cal V}$       & Pretraining                                 & Wikitext, C4                      \\
		Conditional Generation  & $x_{1:N}$                     & $y_{1:M} \in {\cal V}^M$ & Translation,\newline Summarization          & WMT, IWSLT, \newline CNN/DM, XSum \\
		\bottomrule
	\end{tabularx}
	\caption[{Übersicht der verschiedenen \foreigntextquote{english}{Heads} aus {\autocite[Figure 2,top]{1910.03771}}}]{%
		Übersicht der verschiedenen \foreigntextquote{english}{Heads} aus \autocite[Figure 2,top]{1910.03771}.
		Die Token Sequenzen \(x_{1:N}\) stammen hierbei aus einem Vokabular \(\mathcal{V}\),
		während die \(y\) z.B.\, aus einer Menge von Klassen \(\mathcal{C}\) stammen kann.
	}
	\label{tbl:transformers-heads}
\end{table}

Beim \enquote{fine-tuning} werden die einzelnen Parameter des bereits trainierten Netzwerks
für den expliziten Task optimiert. \autocite{towardsdatascience:what-exactly-happens-when-we-fine-tune-bert}
Dies kann durch verschiedene Mechanismen geschehen.
Der einfachste Ansatz sind randomisierte \emph{kleine} Änderungen der Parameter.
Modernere Ansätze gehen hier jedoch gezielter vor
und betrachten z.B.\, Gradienten.
% CHECK Reference AdamW

\subsubsection{Eingaben des Modells}
Mögliche Eingaben in das Modell sind in \Cref{tbl:transformers-heads}
zu sehen.
Die Token \(x_i\) werden durch Zahlen identifiziert
in welche sie von einem \texttt{Tokenizer} umgewandelt werden.
Der \gls{BERT}[-Tokenizer] benutzt sogenannte \foreignquote{english}{sub word}-Token.
Hierbei kann ein Wort aus mehreren Token bestehen.
Die typische \texttt{\#\#ing} des englischen Gerund,
wird hierbei der z.B.\, der Zahl 270 zugeordnet.

Da die tieferen Funktionsweisen von \gls{BERT}
durch die verwendeten Bibliotheken abstrahiert sind,
wird nicht weiter auf die Aufmerksamkeitsmechanismen 
und die Interpretation durch die heads
eingegangen.

\FloatBarrier