Zuerst werden hier die beiden Quellen der Eingaben erklärt.

\subsubsection{Erstellung eines \glspt{gazetteer} aus der \glspt{bllontology}}
Um eine initiale Annotation der Texte durchzuführen,
wird aus der \gls{bllontology} ein \gls{gazetteer}
der Anzeigenamen und der Namen der Objekte erstellt.

Das Format für die \glspl{gazetteer},
welches in \Cref{lst:gazetteerMitURI} zu sehen ist,
besteht aus der Named Entity, einem Tabulator und der \gls{uri},
welche die Named Entity eindeutig bezeichnet.
Der Dateimame in Großbuchstaben
wird als Tag-Typ verwendet.


\begin{listing}
	\begin{tcolorbox}[tlistingstyle]
		\inputminted[
			firstline=1,
			lastline=5,
			fontsize=\footnotesize,
			escapeinside=||,
		]{xml}{../data/gazetteers/bll.de.dict}
	\end{tcolorbox}%
	\nopagebreak
	\caption{Beispielhafte Einträge aus dem \texttt{bll.de.dict} \gls{gazetteer}, welches zusätzlich zum Wort noch eine \gls{uri} für das \gls{namedentitylinking} enthält}
	\label{lst:gazetteerMitURI}
\end{listing}

Für einen vereinfachen Umgang mit den \glspl{gazetteer}
sind in \mintinline{text}{src/oaipmh/gazetteer.py}
Helferfunktionen \mintinline{python}{read_gazetteers},
\mintinline{python}{read_gazetteer} und
\mintinline{python}{write_gazetteer}
definiert,
deren Signatur in \cref{lst:oaipmh:gazetteer:read_gazetteers,lst:oaipmh:gazetteer:read_gazetteer,lst:oaipmh:gazetteer:write_gazetteer}
dargestellt ist. % CHECK Reference

\begin{listing}
	\begin{minipage}{\textwidth}
		\begin{tcolorbox}[tlistingstyle]
			\inputminted[
				firstline=7,
				lastline=9,
				fontsize=\footnotesize,
			]{python}{../src/oaipmh/gazetteer.py}
		\end{tcolorbox}
		\subcaption[a]{\mintinline{python}{read_gazetteers}}% HACK
		\label{lst:oaipmh:gazetteer:read_gazetteers}
	\end{minipage}
	\begin{minipage}{\textwidth}
		\begin{tcolorbox}[tlistingstyle]
			\inputminted[
				firstline=28,
				lastline=28,
				fontsize=\footnotesize,
			]{python}{../src/oaipmh/gazetteer.py}
		\end{tcolorbox}
		\subcaption[b]{\mintinline{python}{read_gazetteer}}% HACK
		\label{lst:oaipmh:gazetteer:read_gazetteer}
	\end{minipage}
	\begin{minipage}{\textwidth}
		\begin{tcolorbox}[tlistingstyle]
			\inputminted[
				firstline=47,
				lastline=47,
				fontsize=\footnotesize,
			]{python}{../src/oaipmh/gazetteer.py}
		\end{tcolorbox}
		\subcaption[c]{\mintinline{python}{write_gazetteer}}% HACK
		\label{lst:oaipmh:gazetteer:write_gazetteer}
	\end{minipage}
	\nopagebreak
	\caption{Signaturen der Helferfunktionen in \mintinline{text}{src/oaipmh/gazetteer.py} für den Umgang mit \glspl{gazetteer}}
	\label{lst:oaipmh:gazetteer}
\end{listing}

\subsubsection{Extraktion der Metadaten}\label{ssec:dataharvesting}
Statt die Metadaten aus dem \gls{fid:linguistik}[-Portal]
zu extrahieren,
werden die Daten mit Hilfe von
\gls{oaipmharvest}
manuell
direkt von den Anbieterseiten extrahiert.
Die so gewonnenen \gls{xml}[-Dateien]
werden im nächsten Schritt weiter bearbeitet.
Durch diesen Weg kann auf die Besonderheiten jedes Anbieters eingegangen werden.


\begin{longlisting}
	\begin{tcolorbox}[tlistingstyle,breakable]
		\inputminted[
			firstline=22,
			lastline=41,
			fontsize=\footnotesize,
			escapeinside=||,
			gobble=10
		]{xml}{listings/oaixml:shortened:2022-07-13__oai_dc__000000000000.xml}
	\end{tcolorbox}%
	\nopagebreak
	\caption{Beispiel Metadaten eines \gls{oai-pmh}[-Eintrags] von \texttt{ubffm} innerhalb des \mintinline{xml}{oai_cd:dc}-Tags}
	\label{lst:oaipmh:xml:record:metadata}
\end{longlisting}

Da das \gls{oai-pmh}[-Protokoll] standardisiert ist,
treten Fallunterscheidungen erst auf,
wenn die eigentlichen Volltexte geladen werden.

Des Weiteren nutzt die \gls{jcs}
auch die \gls{url} der \gls{pdf}[-Dateien]
als \texttt{dc:identifier},
was dazu führt,
dass die Extraktion sehr einfach ist.
Andere Verlage haben die Links zu den eigentlichen \glspl{pdf}
nicht oder zumindest nicht klar erkennbar
in die \texttt{oai:metadata} Einträge ihrer
\texttt{oai:record} codiert.