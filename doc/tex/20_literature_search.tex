Die initiale Recherche zum Thema \gls{namedentityrecognition}
mit \gls{BERT}
lieferte viele verschiedene Ansätze.
Inbesondere wurden viele Quellen zum Thema \gls{distant-supervision}
gefunden.
Die Vorteile von \gls{distant-supervision}
werden in~\ref{ssec:distantlysupervised-namedentityrecognition}
dargestellt
und sollen daher noch nicht im Detail wiedergegeben werden.

\citeauthor{1910.11470} bieten in \citetitle{1910.11470}
einen breiten Überblick über die Möglichkeiten zur \gls{namedentityrecognition}
und die Unterschiede in Abhängigkeit vom Fachgebiet.

In \autocite[Table 1]{1910.11470} ist die Aufteilung am einfachsten zu erkennen
% TODO finish sentence

\begin{figure}[ht]
	% https://latexdraw.com/draw-trees-in-tikz/
	\nocite{latexdraw:trees}
	\begin{tcolorbox}[tfigurestyle]
		\begin{center}
			\begin{tikzpicture}
	[
%		level 0/.style = {red!40!black},
%		level 1/.style = {orange!40!black},
%		level 2/.style = {yellow!40!black},
%		level 3/.style = {green!40!black},
%		level 4/.style = {cyan!40!black},
%		level 5/.style = {purple!40!black},
%		level1/.style = {red!40!black},
%		level2/.style = {orange!40!black},
%		level3/.style = {fill=yellow,draw=yellow!40!black},
%		level4/.style = {green!40!black},
%		level5/.style = {cyan!40!black},
%		level6/.style = {purple!40!black},
		level1/.style = {fill=red},
		level2/.style = {fill=orange},
		level3/.style = {fill=yellow},
		level4/.style = {fill=green},
		level5/.style = {fill=cyan},
		level6/.style = {fill=violet!70!white},
		every node/.append style = {draw, anchor = west},
		grow via three points={one child at (0.5,-0.8) and two children at (0.5,-0.8) and (0.5,-1.6)},
		edge from parent path={(\tikzparentnode\tikzparentanchor) |- (\tikzchildnode\tikzchildanchor)}]
	 
	
	\node[level1] {\gls{machine-learning}}
		child {
			node[level1] {Feature-engineered}
			}
		child {
			node[level2] {Feature-inferring}
			child {
				node[] {word}
			}
			child {
				node[] {character}
			}
			child {
				node[] {word+character}
			}
			child {
				node[] {word+character+affix}
			}
			}
		;
	\end{tikzpicture}
		\end{center}
	\end{tcolorbox}
	\caption[%
        Übersicht der Systemtypen für \glspt{machine-learning}%
    ]{
		Übersicht der Systemtypen für \gls{machine-learning} nach \autocite[Table 1]{1910.11470}\\
        Bei Feature-engineered Systemen werden die Features durch Menschen definiert,
        während Feature-inferring Systeme diese selbstständig aus den Trainingsdaten generieren.
	}%
	\label{fig:model:types}
\end{figure}

Bei den \glslink{neuralnetwork}{neuronalen Netzen} handelte es sich
% TODO finish sentence

% TODO Arras: Kapitel nicht nötig
\FloatBarrier