Verbesserung der eigentlichen Daten ist auch eine Möglichkeit bessere Ergebnisse zu erzielen.

Ein Hinzufügen von Kontext zu den einzelnen klassifizierenden Schritten,
wie in \citetitle{2011.06993v1} beschrieben,
ist vorstellbar.\autocite{2011.06993v1}
Naheliegende Kandidaten wären hier der eigentliche Dokumentenkontext,
aber auch eine Kontextualisierung durch die \mintinline{xml}{dc:subject}-Tags
wäre vorstellbar.

Es könnte auch ein an \gls{ERNIE} 
orientiertes Modell trainiert werden,
welches das Wissen der verschiedenen \glspl{knowledgebase} enthält. \autocite{1905.07129}
Dafür muss jedoch die Rechenleistung entsprechend vorhanden sein:
\foreignblockquote{english}[\autocite{github:thunlp:ERNIE}]{We use 8 NVIDIA-2080Ti to pre-train our model and there are 32 instances in each GPU. It takes nearly one day to finish the training (1 epoch is enough).}
Damit die mit \gls{TransE} erzeugten Einbettungen
kongruent sind,
müssen die Entitäten zusätzlich vereinheitlicht werden
oder das Training auf eine \gls{knowledgebase} beschränkt werden.
Für die Verknüpfung der verschiedenen \glspl{knowledgebase}
könnte,
\href{https://github.com/KRR-Oxford/DeepOnto}{\texttt{DeepOnto}}\footnote{\url{https://github.com/KRR-Oxford/DeepOnto}}
ein hilfreiches Werkzeug für die Identifizierung der Entitäten sein,
wenn die \glspl{knowledgebase} jeweils bereits als \gls{ontology} vorliegen,


Ebsenso könnten Konzepte der \gls{bllontology}
zur Verbesserung der Daten
mit der \gls{Dewey:Decimal:Classification} verknüpft werden.
So hat z.B.\, der Eintrag  aus \cref{lst:ubffm:record} als \texttt{dc:subject} 
den Wert \texttt{dcc:497},
also nach \autocite{oclc:ddc23-summaries} \enquote{North American native languages}
als Thema.

Da sowohl die \gls{bllontology} 
als auch ein Teil der \mintinline{xml}{dc:description} 
bzw.\,der \mintinline{xml}{dc:title} der \gls{oai-pmh}[-Records]
sowohl auf English
als auch auf Deutsch vorliegen,
könnte ein Ansatz,
wie in \citetitle{10.1145/2396761.2398506}
beschrieben,
dafür verwendet werden,
die Trainingsdaten zu verbessern. \autocite{10.1145/2396761.2398506}
Wenn ein Begriff aus der \gls{bllontology} in einer der Sprachen vorkommt,
ist die Wahrscheinlichkeit hoch,
dass er auch in der anderen Sprache vorkommt.
Sollte ein Begiff mit einer \gls{levenshtein_distance} 
unter einer gewissen Schwelle
zu dem erwarteten Begriff
vorkommen,
ist dies ein Kanditat für die Erweiterung des \gls{gazetteer}.
Ein Beispiel ist in \cref{lst:oaipmh:xml:record:title_translation} zu sehen.

\begin{longlisting}
    \begin{tcolorbox}[tlistingstyle]
	\inputminted[
	]{xml}{listings/oai:publikationen.ub.uni-frankfurt.de:6099:shortened.xml}
    \end{tcolorbox}%
    \nopagebreak
	\caption{Beispiel für die Parallelen zwischen den \mintinline{xml}{dc:title} Einträgen
    am Beispiel von \linebreak[2] \mintinline{xml}{<identifier>oai:publikationen.ub.uni-frankfurt.de:6099</identifier>}}
	\label{lst:oaipmh:xml:record:title_translation}
\end{longlisting}

