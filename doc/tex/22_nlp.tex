Bei \gls{naturallanguageprocessing}
geht es um die Verarbeitung und Interpretation von menschlicher Sprache.
Das Ziel ist es Informationen zu gewinnen.

\subsubsection{\glsfmtfull{namedentityrecognition}}
Der Prozess der \gls{namedentityrecognition}
lässt sich nach \autocite{2006.15509} als Lösung des folgenden mathematischen Problems beschreiben:

\begin{prob}[\glsfmtfull{namedentityrecognition}]\label{prob:nlp:ner}
	Für einen \gls{nlp:sentence} \glsentryformula{nlp:sentence}
	wird eine \gls{nlp:label_sequence} \glsentryformula{nlp:label_sequence} gesucht.
	Die \texttt{BIO}-Klassifikation von \citeauthor{10.1145/2396761.2398506}
	\autocite{10.1145/2396761.2398506}
	fordert hierbei,
	dass \(
	\glssymbol{nlp:label}_i \in
	\left\lbrace
	\glssymbol{nlp:label:beginning}\texttt{-X},
	\glssymbol{nlp:label:inside}\texttt{-X},
	\glssymbol{nlp:label:outside}
	:
	\text{\texttt{X} ist Entitätstyp}
	\right\rbrace
	\)
	und dass gilt:
	\[
		\glssymbol{nlp:label}_i =
		\begin{cases}
			\texttt{\glssymbol{nlp:label:beginning}-X},
			\text{wenn \(\glssymbol{nlp:token}_i\)
				der Beginn einer Entität vom Typ \(\texttt{X}\) ist
			} \\
			\texttt{\glssymbol{nlp:label:inside}-X},
			\text{wenn \(\glssymbol{nlp:token}_i\)
				ein Token einer Entität vom Typ \(\texttt{X}\)
				und nicht deren Beginn ist
			} \\
			\glssymbol{nlp:label:outside}, \text{sonst}.
		\end{cases}
	\]
	Da die \glspl{nlp:sentence} meistens in Tokens aufgeteilt werden,
	spricht man auch von einer Tokenklassifizierung-Aufgabe.
\end{prob}

Der klassische Ansatz zur Erzeugung einer \gls{nlp:label_sequence} ist es,
die Eingabe in mehreren Schritten die Eingabe um Informationen (Features) anzureichern.
Diese werden in den jeweils darauf folgenden Schritten, unterstützend eingesetzt.
Dazu gehören z.B.\, \textquote[{\autocite[§19]{fortext-2018-id-36}}]{häufig vorher genannte Wörter (sowie z.B. bei Orten das Wort „in"),
	Darstellungsformat (bei Daten so etwas wie Zahl Monat Jahr),
	Groß- und Kleinschreibung} oder auch
die \textquote[{\autocite[§19]{fortext-2018-id-36}}]{Position im Satz}. % CHECK how this looks
Eine Darstellung einer beispielhaften \gls{naturallanguageprocessing}[-Architektur] ist in \Cref{fig:corenlp:architecture}
zu sehen.

% https://tex.stackexchange.com/questions/83069/how-to-stack-two-subfigures-next-to-a-third-subfigure
\nocite{stack:tex:two-subfigures-next-to-a-third-subfigure}
\begin{figure}
	\begin{subfigure}[b]{0.45\textwidth}
		\begin{tcolorbox}[tpipelinestyle]
			\begin{adjustbox}{center,max width=0.9\linewidth,max totalheight=0.9\textheight}
				\InputIfExists{figures/pipeline:corenlp.latex}
			\end{adjustbox}
		\end{tcolorbox}
		\caption{Darstellung der Architektur von Stanford CoreNLP \autocite[Figure 1]{P14-5010}}
		\label{fig:corenlp:architecture}
	\end{subfigure}
	\hfill
	\begin{subfigure}[b]{0.45\textwidth}
		\begin{tcolorbox}[tpipelinestyle]
			\begin{adjustbox}{center,max width=\linewidth,max totalheight=0.9\textheight}
				\InputIfExists{figures/pipeline:nlp:machinelearning.latex}
			\end{adjustbox}
		\end{tcolorbox}
		\caption{Aufbau der Pipeline für die Datenverarbeitung}
		\label{fig:pipeline:nlp:machinelearning}
		%\end{subfigure}

		\vfill

		%\begin{subfigure}[b]{0.45\textwidth}
		\begin{tcolorbox}[tpipelinestyle]
			\begin{adjustbox}{center,max width=\linewidth,max totalheight=0.9\textheight}
				\InputIfExists{figures/huggingface:model.latex}
			\end{adjustbox}
		\end{tcolorbox}
		\caption[Aufbau eines {\gls{transformer}[ Modells]}]{Aufbau eines \gls{transformer}[ Modells]:
			Der Tokenizer übersetzt die Eingabe in dünnbesetzte Vektoren über einer entsprechenden Basis.
			Diese Vektoren bzw.\, Tensoren werden dann durch den eigentlichen \gls{transformer}
			in eine kontextuelle Einbettung
			überführt.
			Diese wird dann durch den Head auf eine aufgabenspezifische Vorhersage umgerechnet.
		}
		\label{fig:huggingface:transformer}
	\end{subfigure}
	\caption{Darstellung von Architekturen und Pipelines}
\end{figure}

Ein gängiger Aufbau für eine \gls{naturallanguageprocessing}
Pipeline
auf Basis von \glslink{machine-learning}{maschinellem Lernen}
ist nach
\autocite{1910.03771}
der in \Cref{fig:pipeline:nlp:machinelearning} dargestellte.
Hierbei werden keine weiteren Features definiert
und das System muss diese selber erkennen.
\citeauthor{1905.05950} stellen in \citetitle{1905.05950} jedoch fest,
dass auch \gls{BERT} teilweise ähnliche Zwischenschritte
wie in der klassischen \gls{naturallanguageprocessing}[-Pipeline]
abbildet.\autocite{1905.05950}

In der Regel besteht ein solches \gls{naturallanguageprocessing}[-Modell] aus drei Stufen,
wie in \Cref{fig:huggingface:transformer} dargestellt ist.
Der Tokenizer wandelt den Text in eine numerische Repräsentation um.
Diese wird durch den \gls{transformer} in einen Tensor aus Wahrscheinlichkeiten umgerechnet.
Der head erstellt aus diesem Wahrscheinlichkeitstensor eine Interpretation für die Aufgabe.


Auf der Ebene des Tokenizers gibt es Unterschiede.
In \autocite[Table 1]{1910.11470} sind diese einfach zu erkennen,


\begin{figure}[ht]
	% https://latexdraw.com/draw-trees-in-tikz/
	\nocite{latexdraw:trees}
	\begin{tcolorbox}[tfigurestyle]
		\begin{center}
			\begin{tikzpicture}
	[
%		level 0/.style = {red!40!black},
%		level 1/.style = {orange!40!black},
%		level 2/.style = {yellow!40!black},
%		level 3/.style = {green!40!black},
%		level 4/.style = {cyan!40!black},
%		level 5/.style = {purple!40!black},
%		level1/.style = {red!40!black},
%		level2/.style = {orange!40!black},
%		level3/.style = {fill=yellow,draw=yellow!40!black},
%		level4/.style = {green!40!black},
%		level5/.style = {cyan!40!black},
%		level6/.style = {purple!40!black},
		level1/.style = {fill=red},
		level2/.style = {fill=orange},
		level3/.style = {fill=yellow},
		level4/.style = {fill=green},
		level5/.style = {fill=cyan},
		level6/.style = {fill=violet!70!white},
		every node/.append style = {draw, anchor = west},
		grow via three points={one child at (0.5,-0.8) and two children at (0.5,-0.8) and (0.5,-1.6)},
		edge from parent path={(\tikzparentnode\tikzparentanchor) |- (\tikzchildnode\tikzchildanchor)}]
	 
	
	\node[level1] {\gls{machine-learning}}
		child {
			node[level1] {Feature-engineered}
			}
		child {
			node[level2] {Feature-inferring}
			child {
				node[] {word}
			}
			child {
				node[] {character}
			}
			child {
				node[] {word+character}
			}
			child {
				node[] {word+character+affix}
			}
			}
		;
	\end{tikzpicture}
		\end{center}
	\end{tcolorbox}
	\caption[%
        Übersicht der Systemtypen für \glspt{machine-learning}%
    ]{
		Übersicht der Systemtypen für \gls{machine-learning} im Bereich \gls{naturallanguageprocessing}
		nach \autocite[Table 1]{1910.11470}\\
        Bei Feature-engineered Systemen werden die Features durch Menschen definiert,
        während Feature-inferring Systeme diese selbstständig aus den Trainingsdaten generieren.
	}%
	\label{fig:model:types}
\end{figure}


\begin{defn}[Güte eines Modells]
	Seien \(M\) bereits annotierte \glslink{nlp:sentence}{Sätze}
	\glsentryformula{nlp:labelled_sentences} gegeben
	und beschreibe \glsentryformula{nlp:model} ein \gls{namedentityrecognition}[-Modell],
	wobei \glssymbol{nlp:sentence} ein Satz
	und \glssymbol{nlp:model:parameters} die Parameter des Modells sind.
	Dann lässt sich anhand von \Cref{prob:nlp:ner}
	die Güte des Modells über eine Metrik bestimmen.
\end{defn}

Eine beliebte Metrik im Rahmen des \gls{machine-learning} ist die \gls{crossentropyloss}[-Funktion],
wie sie in \autocite[Abschnitt 5.5]{jurafsky2000speech} definiert wird:
\glsentryformula{crossentropyloss}.
Hierbei wäre der ideale Fall,
dass jeder Satz
\glspl{nlp:label} entsprechend der vorgegeben (idealen) Annotation
zugewiesen bekommt.

\begin{thm}
	Dementsprechend werden die \gls{nlp:model:parameters} so gesucht,
	dass sie diese Distanz über alle Trainingsdaten minimieren.
	Somit sind die idealen Parameter für die \(M\) \glslink{nlp:sentence}{Sätze}
	durch die Formel
	\autocite[1]{2006.15509}
	gegeben:
	\begin{equation}
		\hat{\glssymbol{nlp:model:parameters}} : =
		\argmin_{\glssymbol{nlp:model:parameters}}
		\frac{1}{M}
		\sum_{m=1}^{M}
		\glssymbol{crossentropyloss}\left(
		\glssymbol{nlp:label_sequence}_m,
		\glssymbol{nlp:model}\left(
			\glssymbol{nlp:sentence}_m;
			\glssymbol{nlp:model:parameters}
			\right)
		\right)
	\end{equation}
\end{thm}

\subsubsection{\glsfmtfull{namedentitylinking}}
Das Ziel von \gls{namedentitylinking} ist es
die die per \gls{namedentityrecognition} erkannten Entitäten
mit einer \gls{knowledgebase} zuverknüpfen.
Dies lässt sich als folgendes Problem darstellen.

\begin{prob}[\glsfmtfull{namedentitylinking}]\label{prob:nlp:nel}
	Für einen \gls{nlp:sentence} \glsentryformula{nlp:sentence}
	wird eine \gls{nlp:label_sequence} \glsentryformula{nlp:label_sequence} gesucht,
	sodass \(
	\glssymbol{nlp:label}_i \in
	\left\lbrace
	\text{Entitätsreferenz}
	\right\rbrace
	\cup
	\left\lbrace
	\text{NONE}
	\right\rbrace
	\).
\end{prob}

Da es oftmals mehrere Kandidaten  für die Verlinkung in der \gls{knowledgebase} gibt,
wird im Schritt der \gls{namedentitydisambiguation} eine ausgewählt.