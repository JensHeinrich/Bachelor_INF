Im Folgenden werden die zwei Datenquellen,
die für diese Arbeit verwendet wurden,
beschrieben.
Die erste Datenquelle ist die \gls{bllontology},
die für die Erstellung der Annotationen benötigt wird.
Danach folgt das \glsfmtfull{oai-pmh},
welches die zu annotierenden Daten liefert.

\subsubsection{\glspt{bllontology}}


\begin{defn}[Thesaurus\autocite{Arano:ThesaurusAndOntologies}] 
	Ein Thesaurus 
	ist eine Art von dokumentierender
	Sprache,
	die die konzeptuelle Struktur des Wissens eines bestimmten Fachgebietes
	darstellt.
	Er stellt den semantischen Zusammenhang zwischen verschieden Konzepten
	durch ein eingeschränktes Vokabular für die Beziehungen
	dar.
\end{defn}

Das Ziel eines Thesaurus ist \enquote{synonymy},
also die Überlappung der Bedeutung von Wörtern \autocite{oxfordbibliographies:Synonymy},
und \enquote{polysemy},
also die Mehrdeutigkeit der Bedeutung innerhalb eines Bereiches \autocite{oxfordbibliographies:Polysemy},
zu reduzieren.

Im Folgenden werden Beispiele von Dateien in der \gls{turtle}[ Syntax] gezeigt \autocite{w3c:turtle}.
Diese wird daher kurz erklärt.

Die Syntax besteht im Grunde nur aus Tripeln von \emph{Subjekt}, \emph{Prädikat} und \emph{Objekt}.
Zur Vereinfachung bleibt das aktuelle \emph{Subjekt} bis zum nächsten \emph{Punkt (.)}
und das aktuelle \emph{Prädikat} bis zum nächsten \emph{Semikolon (;)} aktiv.
Ein \emph{Komma (,)} trennt in diesem Zusammenhang die \emph{Objekte}.
Bei StringLiterals 
ist eine \gls{i18n}
möglich.
\mintinline{turtle}{"TEXT"@de} ist somit als String für die Sprache Deutsch definiert.

Damit die Konzepte eineindeutig sind,
werden sie über \gls{uri} beschrieben;
zur Vereinfachung können diese aber durch global erklärte Präfixe ersetzt werden.
Ein Beispiel hierfür ist in \Cref{lst:bll:thesaurus:prefixes}.

\begin{listing}
	\begin{tcolorbox}[tlistingstyle]
		\inputminted[
			firstline=1,
			lastline=13
		]{turtle}{../data/bll-thesaurus.ttl}
	\end{tcolorbox}
	\caption{Erklärung der globalen Präfixe in der \texttt{bll-thesaurus.ttl}}
	\label{lst:bll:thesaurus:prefixes}
\end{listing}

Nach dem die Präfixe
erklärt sind,
sind die zwei beispielhaften Einträge in \Cref{lst:bll:thesaurus:seneca} eindeutig definiert.
Der Eintrag \mintinline{turtle}{bllt:bll-133103862}
ist eine Klasse und ein Konzept.
Er wird durch \mintinline{turtle}{"Seneca"}
sowohl auf Deutsch,
als auch auf Englisch
und sowohl für das Konzept,
wie auch für die Anzeige klassifiziert.
Er ist eine Unterklasse von \mintinline{turtle}{bllt:BLLConcept}.
Als gröbere Klassifizierung gilt \mintinline{turtle}{bllt:bll-133108791} (Irokesisch).
Die Notation \mintinline{turtle}{"02.25.01.047.06"} repräsentiert diese Hierarchie,
wie in \Cref{fig:bll:notation:seneca}
zu sehen ist.

\begin{listing}[ht]
	\begin{tcolorbox}[tlistingstyle]
		\inputminted[
			firstline=12171,
			lastline=12183
		]{turtle}{../data/bll-thesaurus.ttl}
	\end{tcolorbox}
	\caption{Beispieleinträge aus \texttt{bll-thesaurus.ttl}}
	\label{lst:bll:thesaurus:seneca}
\end{listing}

\begin{figure}[ht]
	% https://latexdraw.com/draw-trees-in-tikz/
	\nocite{latexdraw:trees}
	\begin{tcolorbox}[tfigurestyle]
		\begin{center}
			\begin{tikzpicture}
	[
		level1/.style = {fill=red},
		level2/.style = {fill=orange},
		level3/.style = {fill=yellow},
		level4/.style = {fill=green},
		level5/.style = {fill=cyan},
		level6/.style = {fill=violet!70!white},
		every node/.append style = {draw, anchor = west},
		grow via three points={one child at (0.5,-0.8) and two children at (0.5,-0.8) and (0.5,-1.6)},
		edge from parent path={(\tikzparentnode\tikzparentanchor) |- (\tikzchildnode\tikzchildanchor)}]
	 
	
	\node[level1] {00}
		child {node[level1] {BLL-Klassifikation} edge from parent [dashed]}
		child {
			node[level2] {02}
			child {node[level2] {Nicht-indoeuropäische Sprachen} edge from parent [dashed]}
			child {
				node[level3] {02.25}
				child {node[level3] {Indigene amerikanische Sprachen} edge from parent [dashed]}
				child {
					node[level4] {02.25.01}
					child {node[level4] {Indigene Sprachen Nordamerikas und Zentralamerikas} edge from parent [dashed]}
					child {
						node[level5] {02.25.01.047}
						child {node[level5] {Irokesisch} edge from parent [dashed]}
						child {
							node[level6] {02.25.01.047.06 }
							child {node[level6] {Seneca} edge from parent [dashed]}
						}
					}
				}
			}
		};
	\end{tikzpicture}
		\end{center}
	\end{tcolorbox}
	\caption{Darstellung der Notation des Eintrages \Cref{lst:bll:thesaurus:seneca} als Baum mit den übergeordneten Klassen}%
	\label{fig:bll:notation:seneca}
\end{figure}

\FloatBarrier
\begin{defn}[\glspt{ontology}\autocite{10.1016/S0169-023X:97:00056-6}\autocite{Arano:ThesaurusAndOntologies}]
	\foreignblockquote{english}[{\autocite[Abschnitt 1.]{10.1006/knac.1993.1008}  und \autocite[Abschnitt 2.1]{Borst1997} via \autocite[Abschnitt 6.1]{10.1016/S0169-023X:97:00056-6}} ]{An ontology is a formal, explicit specification of a shared conceptualisation}
	wobei die folgenden Definitionen gelten:

	\begin{enumerate}
		\item Konzeptualisierung
		      : Ein abstraktes Modell eines Phänomens der realen Welt basierend auf den relevanten Konzepten des Phänomens
		\item Explizit: der Typ des Konzepts und die Einschränkungen seiner Nutzing sind explizit definiert
		\item Formal: Die Syntax ist präzise genug, um von einem Computer verstanden zu werden
		\item Geteilt: das Wissen ist von einer Gruppe akzeptiert
	\end{enumerate}

	Die Konzeptualisierung
	wird von \citeauthor{Arano:ThesaurusAndOntologies}
	noch weiter spezifiert,
	sodass sie eine Perspektive eine bestimmten Realität involvieren muss
	und diese auf der konzeptuellen Struktur einer \gls{knowledgebase}
	begründet wird.

	Das Ziel einer \gls{ontology}
	ist das Teilen des Wissens,
	welches sie repräsentiert.
\end{defn}
\FloatBarrier

Da sich die Präfixe in der \gls{ontology}
von denen im Thesaurus 
unterscheiden,
werden sie in \Cref{lst:bll:ontoloy:prefixes} noch einmal explizit dargestellt.

\begin{listing}
	\begin{tcolorbox}[tlistingstyle]
		\inputminted[
			firstline=1,
			lastline=6
		]{turtle}{listings/turtle:shortened:bll-ontology.ttl}
	\end{tcolorbox}
	\caption{Erklärung der globalen Präfixe in der \texttt{bll-ontology.ttl}}
	\label{lst:bll:ontoloy:prefixes}
\end{listing}

Nachdem die Präfixe
erklärt sind,
sind die zwei beispielhaften Einträge in \Cref{lst:bll:ontology:seneca} eindeutig definiert.
Das erste Beispiel ist wieder \mintinline{turtle}{"Seneca"}.
Dieses Mal ist die Klassifizierung als \enquote{\mintinline{turtle}{a skos:Concept}} % CHECK linebreak
nicht mehr enthalten,
dafür ist die neue Klassifikation 
als \enquote{\mintinline{turtle}{rdfs:subClassOf bllt:NorthernIroquoian}} % CHECK format
hinzugekommenen.
Das Format dieser \enquote{Elternklasse} zeigt an,
dass es eine manuell angelegte Klasse ist,
da der Name innerhalb des \texttt{bllt}-Namespaces keine Nummer ist.
Das zweite Beispiel zeigt zusätzliche Prädikate in der \gls{owl}[-Syntax],
wie Versionierungsinformationen und Äquivalenz.

\begin{listing}
	\begin{tcolorbox}[tlistingstyle]
		\inputminted[
			firstline=8,
			lastline=25
		]{turtle}{listings/turtle:shortened:bll-ontology.ttl}
	\end{tcolorbox}
	\caption{Leicht umformatierte Beispieleinträge aus \texttt{bll-ontology.ttl}
		(Der Zeilenumbruch nach dem Subjekt ist jeweils entfernt,
		damit das Syntaxhighlighting mit \gls{pygments} funktioniert)
	}
	\label{lst:bll:ontology:seneca}
\end{listing}


\begin{defn}[\glsfmtfull{bll}]
	Die \gls{bll} ist eine Bibliographie,
	also ein Verzeichnis von Literaturnachweisen
	zu linguistischer Literatur,
	insbesondere der Allgemeinen Linguistik
	und \textquote[\autocite{linguistik:de:kataloge:info}]{der anglistischen, germanistischen und romanistischen Sprachwissenschaft}.
\end{defn}

Zu dem oben genannten Verzeichnis existiert der \gls{bll}[ Thesaurus],
welcher für die Indizierung im Portal verwendet wird.
Bei der Umwandelung des \gls{bll}[ Thesaurus] in die \gls{bllontology},
die initial von \citeauthor{L16-1707} durchgeführt wird \autocite{L16-1707},
wird der Umfang an Begriffen eingeschränkt
auf Konzepte aus den Zweigen \foreigntextquote{english}[\autocite{data:linguistic:ontology-doc}]{Syntax, Morphology, Lexicology and Phonology}
und einigen zusätzlichen Einträgen aus den Zweigen \foreigntextquote{english}[\autocite{data:linguistic:ontology-doc}]{Graphemics and Semantics}.
Daher ist z.B.\,  der Term \texttt{bllt:bll-467296421} aus \Cref{lst:bll:thesaurus:seneca}
nicht in der \gls{bllontology} verzeichnet.

\subsubsection{\glsfmtfull{oai-pmh}}
Die \glsfmtfull{openarchivesinitiative} hat mit \gls{oai-pmh}
einen Standard für Interoperabilität zwischen verschiedenen Metadaten-Quellen
und Dienstanbietern definiert.
In \Cref{lst:oaipmh:xml:begin} ist der Anfang einer mit \gls{oaipmharvest} erzeugten
\gls{xml}[-Datei] dargestellt.

\begin{longlisting}
	\begin{tcolorbox}[tlistingstyle]
		\inputminted[
			firstline=1,
			lastline=21,
			fontsize=\footnotesize,
			escapeinside=||,
			gobble=10
		]{xml}{listings/oaixml:shortened:2022-07-13__oai_dc__000000000000.xml}
	\end{tcolorbox}
	\nopagebreak
	\caption{Beispiel Beginn eines \gls{oai-pmh}[-Eintrags]}
	\label{lst:oaipmh:xml:begin}
\end{longlisting}

Da der Inhalt innerhalb des \mintinline{xml}{oai_dc:dc}-Tags
durch das \gls{dublin-core}\footnote{\url{https://www.dublincore.org/specifications/dublin-core/dces/1999-07-02/}}
festgelegt ist,
liegt hier der Fokus.
Durch den Standard sind
der Titel (\mintinline{xml}{dc:title}),
das Thema (\mintinline{xml}{dc:subject}), welches oft als Liste von Schlagworten ausgedrückt wird,
Beschreibungen (\mintinline{xml}{dc:description})
und
eindeutige Referenzen (\mintinline{xml}{dc:identifier}) auf die Ressource
als Einträge definiert.
Auch die verwandten Ressourcen (\mintinline{xml}{dc:relation})
und das Format (\mintinline{xml}{dc:format})
sind enthalten.

Wenn der Volltext benötigt wird,
müssen die \mintinline{xml}{dc:identifier} oder die \mintinline{xml}{dc:relation}-Einträge
betrachtet werden.
Je nach Anbieter können hier bereits Referenzen auf Volltext-Dokumente hinterlegt sein
oder die Referenzen weisen nur auf Webseiten,
auf welchen die Referenzen auf die Volltext-Dokumente gefunden werden müssen.
