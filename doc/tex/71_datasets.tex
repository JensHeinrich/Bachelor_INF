Hier werden Verbesserungen vorgestellt,
die an der \gls{huggingface:datasets}[-Klasse]
vorgenommen werden können.

Durch die direkte Nutzung der \gls{api}
statt der durch \gls{oaipmharvest} generierten \gls{xml}[-Dateien]
könnte der Prozess vereinfacht werden.
Der \gls{oai-pmh}[-Standard] bietet die Möglichkeit Treffer nach Datum einzuschränken.
Die kontinuierliche Verarbeitung,
der seit dem letzen Verarbeitungslauf hinzugekommenen Einträge,
würde somit vereinfacht.
Auch der zusätzlicher Schritt der Nutzung von \gls{oaipmharvest} würde wegfallen.

In dem Zuge wäre auch die Umstellung auf Streaming denkbar.
Das so erstellte \mintinline{python}{IterableDataset}
lädt neue Daten stückweise,
während auf die Einträge zugegriffen wird.
Dadurch wird der Speicherbedarf reduziert
und der Ladevorgang beschleunigt.

Ein alternativer Ansatze wäre,
anstatt die Einträge mit der Entität zu labeln,
eine Liste von Labeln zu nutzen,
ähnlich zum Ansatz von \autocite{2006.15509}.

Das Stringmatching in \cref{lst:stringmatching} könnte stark verbessert werden
durch die Verwendung einer besseren Datenstruktur.
Eine Struktur die sinnvoll erscheint,
wäre der Präfixbaum (auch Trie genannt).

Alternativ würde sich die Verwendung von \gls{multilayerfinitestatetransducer},
wie in \citetitle{OASIcs-LDK-2019-11}
beschrieben,
anbieten.\autocite{OASIcs-LDK-2019-11}
Hierfür werden verschiedene \gls{finitestatetransducer}
hintereinander kaskadiert;
dieser Ansatz beruht auf \citetitle{10.1017/S1351324997001599}
von \citeauthor{10.1017/S1351324997001599} \autocite{10.1017/S1351324997001599}
mit dem Ziel keine vollständige Liste
aller möglichen Kombinationen von Fällen,
sondern nur solcher,
die auch grammatikalisch vorkommen können,
zu erstellen.
