\subsubsection{Vorbereitung der Eingaben}

Zuerst wird der Text in Tokens aufgeteilt,
welche in numerische Werte übersetzt werden können.
Diese numerischen Werte können durch das neuronale Netzwerk verarbeitet werden,
da dieses aber eine Eingabe fixer Länge erwartet,
müssen zu lange Sequenzen abgeschnitten (\enquote{truncate})
und zu kurze Sequenzen ergänzt (\enquote{pad}) werden.
Die so bearbeiteten Werte werden anschießend in einen \gls{tensor} umgewandelt,
mit dem die weiteren Berechnungen durchgeführt werden können.
Diese werden wird von einem \enquote{preprocessor}
bzw.\, \enquote{tokenizer} ausgeführt.
\autocite{huggingface:docs:Transformers:preprocessing}

\subsubsection{Training}

Da potentiell die Grenzen der so erzeugten Token
nicht mit den Grenzen der Token des \gls{huggingface:datasets}
übereinstimmen,
müssen die Annotationen angepasst werden.
Die \mintinline{python}{tokenize_and_align_labels} Funktion,
die in \Cref{lst:train} dargestellt ist,
iteriert über die Token
und weisst dem jeweils ersten Token eines Worts die Annotation des Worts zu.\autocite{huggingface:course:chapter7:2}


\begin{listing}% CHECK longlisting
	\begin{tcolorbox}[tlistingstyle,breakable]
		\inputminted[
			fontsize=\footnotesize
		]{python}{../src/train.py}
	\end{tcolorbox}
	\caption{Code zum Trainieren eines Modells \texttt{src/train.py}}
	\label{lst:train}
\end{listing}
