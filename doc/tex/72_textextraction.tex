Die eigentliche Textextraktion 
könnte durch Wechsel auf eine \gls{ocr}[-basierende] Methode
verbessert werden.
Diese bringt jedoch auch Schwierigkeiten mit sich, 
wie schon die Studie \citetitle{SBB:OCRStudie} aus dem Jahr \citeyear{SBB:OCRStudie}
von \citeauthor{SBB:OCRStudie} in \autocite[6.2]{SBB:OCRStudie} aufzeigt.
Hier wurden teilweise weniger als \(90\%\) Erkennungsgüte erreicht.
Dennoch wäre eine Verbindung von \gls{ocr-d} mit den semantischen Annotationen
vorstellbar.
Aber auch eine Verbesserung von \gls{pypdf2}
durch die Implementierung der erweiterten Encodings \texttt{/B5pc-H} und \texttt{/90msp-RKSJ-H} 
wäre denkbar.
Die lösten bei der Entwicklung \mintinline{python}{NotImplementedError}s aus.   