\glsfmtfull{BERT} ist ein Modell für \gls{naturallanguageprocessing}.
Es wurde von \citeauthor{1810.04805} auf Basis von \gls{transformer} entworfen,
Diese Architektur beruht vor allem auf Aufmerksamkeitsmechanismen,
welche in \autocite{1706.03762} vorgestellt wurde.
Der Vorteil dieser Architektur bei \gls{naturallanguageprocessing}
gegenüber den vorherigen \gls{lstm}-basierenden Architekturen ist es,
dass beidseitiger Kontext für die Bewertung der Eingaben herangezogen werden kann.

\subsubsection{Training des Modells}
Das Modell wird auf \texttt{unlabeled data}
mit zwei verschiedenen Aufgaben trainiert.
Beim \gls{nlp:task:mlm}
werden randomisiert 15\% der Token maskiert.
Da in späteren Aufgaben keine \texttt{[MASK]}
Token mehr auftauchen,
werden 10\% der Token durch ein anderes Token
und 10\% der Token durch sich selbst \enquote{maskiert}.
\autocite[3.1 Task \# 1]{1810.04805}.
Dieser Task ermöglicht dem Modell
eine kontextuelles \enquote{Verständnis}
einzelner Wörter bzw. Token im Satz zu erlernen.

Die zweite Aufgabe ist \gls{nlp:task:nsp}.
Hierbei werden zwei Sätze A und B aus dem Corpus gewählt,
sodass in 50\% der Fälle Satz B auf Satz A folgt
und in 50\% der Fälle B ein zufälliger Satz aus dem Corpus ist.
\autocite[3.1 Task \# 2]{1810.04805}.
Die kontextuelle Einbettung des  \texttt{[CLS]} Tokens
wird hierbei für die Auswertung betrachtet.
Das Modell lernt somit Zusammenhänge zwischen Sätzen.

Insgesamt entwickelt das Modell eine Darstellung der Informationen,
wie sie in~\Cref{ssec:dartstellbarkeit}
beschrieben wird.
Diese Darstellung wird von den \texttt{heads} verwendet,
um die eigentlichen Aufgabenstellungen zu lösen.
Verschiedenen \texttt{heads} sind am Beispiel der \gls{huggingface:transformers}[-Bibliothek]
in \Cref{tbl:transformers-heads} aufgelistet.

\begin{table}
	\begin{tabularx}{\linewidth}{lccp{3cm}p{3cm}}
		\toprule
		\multicolumn{5}{c}{\textbf{Heads}}                                                                                                                                   \\
		Name                    & Input                         & Output                   & Tasks                                       & Ex. Datasets                      \\
		\midrule
		Language Modeling       & $x_{1:n-1}$                   & $x_n \in {\cal V}$       & Generation                                  & WikiText-103                      \\
		Sequence Classification & $x_{1:N}$                     & $y \in {\cal C}$         & Classification, \newline Sentiment Analysis & GLUE, SST, \newline MNLI          \\
		Question Answering      & $x_{1:M},$ \newline $x_{M:N}$ & $y$ span $[1:N]$         & QA,  Reading\newline Comprehension          & SQuAD, \newline Natural Questions \\
		Token Classification    & $x_{1:N}$                     & $y_{1:N} \in {\cal C}^N$ & NER, Tagging                                & OntoNotes, WNUT                   \\
		Multiple Choice         & $x_{1:N}, {\cal X}$           & $y \in {\cal X}$         & Text Selection                              & SWAG, ARC                         \\
		Masked LM               & $x_{1:N\setminus n}$          & $x_n \in {\cal V}$       & Pretraining                                 & Wikitext, C4                      \\
		Conditional Generation  & $x_{1:N}$                     & $y_{1:M} \in {\cal V}^M$ & Translation,\newline Summarization          & WMT, IWSLT, \newline CNN/DM, XSum \\
		\bottomrule
	\end{tabularx}
	\caption[{Übersicht der verschiedenen \foreigntextquote{english}{Heads} aus {\autocite[Figure 2,top]{1910.03771}}}]{%
		Übersicht der verschiedenen \foreigntextquote{english}{Heads} aus \autocite[Figure 2,top]{1910.03771}.
		Die Token Sequenzen \(x_{1:N}\) stammen hierbei aus einem Vokabular \(\mathcal{V}\),
		während die \(y\) z.B.\, aus einer Menge von Klassen \(\mathcal{C}\) stammen kann.
	}
	\label{tbl:transformers-heads}
\end{table}

Beim \enquote{fine-tuning} werden die einzelnen Parameter des bereits trainierten Netzwerks
für den expliziten Task optimiert. \autocite{towardsdatascience:what-exactly-happens-when-we-fine-tune-bert}
Dies kann durch verschiedene Mechanismen geschehen.
Der einfachste Ansatz sind randomisierte \emph{kleine} Änderungen der Parameter.
Modernere Ansätze gehen hier jedoch gezielter vor
und betrachten z.B.\, Gradienten.
% CHECK Reference AdamW

\subsubsection{Eingaben des Modells}
Mögliche Eingaben in das Modell sind in \Cref{tbl:transformers-heads}
zu sehen.
Die Token \(x_i\) werden durch Zahlen identifiziert
in welche sie von einem \texttt{Tokenizer} umgewandelt werden.
Der \gls{BERT}[-Tokenizer] benutzt sogenannte \foreignquote{english}{sub word}-Token.
Hierbei kann ein Wort aus mehreren Token bestehen.
Die typische \texttt{\#\#ing} des englischen Gerund,
wird hierbei der z.B.\, der Zahl 270 zugeordnet.

% TODO

Die Eingabe in das Netzwerk besteht aus den Tokens Tokens11 ... Tokens1N
bzw. Tokens21 ... Tokens2M,
welche den Text repräsentieren,
und den folgenden \enquote{technischen} Token:
CLS, SEP und dem optionale [MASK] Token.

Eine gute Übersicht,
wie die Eingabe in ein solches Netzwerk aussehen kann,
ist in \Cref{tbl:transformers-heads} zu sehen.


