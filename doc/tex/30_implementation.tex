
Die Verarbeitung der verschiedenen Eingaben
geschieht für das Training in der in \Cref{fig:pipeline:training}
dargestellten Pipeline,
während die Verarbeitung der produktiven Daten in einer Pipeline,
wie sie in \Cref{fig:pipeline:production} dargestellt ist,
erfolgen würde.

\begin{figure}
	% \begin{tcolorbox}[tpipelinestyle]
		\begin{adjustbox}{center,max width=\textwidth,max totalheight=0.9\textheight}
			\InputIfExists{figures/pipeline:training.latex}
		\end{adjustbox}
	% \end{tcolorbox}
	\caption{Aufbau der Pipeline für das Training des Modells}
	\label{fig:pipeline:training}
\end{figure}

\begin{figure}
	% \begin{tcolorbox}[tpipelinestyle]
		\begin{adjustbox}{pagecenter,max width=\textwidth,max totalheight=0.9\textheight}
			\InputIfExists{figures/pipeline:production.latex}
		\end{adjustbox}
	% \end{tcolorbox}
	\caption{Aufbau der Pipeline für die Datenverarbeitung}
	\label{fig:pipeline:production}
\end{figure}

Im Folgenden werden die einzelnen Abschnitte der Implementierung näher erläutert.

\subsection{Datenquellen}
Zuerst werden hier die beiden Quellen der Eingaben erklärt.

\subsubsection{Erstellung eines \glspt{gazetteer} aus der \glspt{bllontology}}
Um eine initiale Annotation der Texte durchzuführen,
wird aus der \gls{bllontology} ein \gls{gazetteer}
der Anzeigenamen und der Namen der Objekte erstellt.

Das Format für die \glspl{gazetteer},
welches in \Cref{lst:gazetteerMitURI} zu sehen ist,
besteht aus der Named Entity, einem Tabulator und der \gls{uri},
welche die Named Entity eindeutig bezeichnet.
Der Dateimame in Großbuchstaben
wird als Tag-Typ verwendet.


\begin{listing}
	\begin{tcolorbox}[tlistingstyle]
		\inputminted[
			firstline=1,
			lastline=5,
			fontsize=\footnotesize,
			escapeinside=||,
		]{xml}{../data/gazetteers/bll.de.dict}
	\end{tcolorbox}%
	\nopagebreak
	\caption{Beispielhafte Einträge aus dem \texttt{bll.de.dict} \gls{gazetteer}, welches zusätzlich zum Wort noch eine \gls{uri} für das \gls{namedentitylinking} enthält}
	\label{lst:gazetteerMitURI}
\end{listing}

Für einen vereinfachen Umgang mit den \glspl{gazetteer}
sind in \mintinline{text}{src/oaipmh/gazetteer.py}
Helferfunktionen \mintinline{python}{read_gazetteers},
\mintinline{python}{read_gazetteer} und
\mintinline{python}{write_gazetteer}
definiert,
deren Signatur in \cref{lst:oaipmh:gazetteer:read_gazetteers,lst:oaipmh:gazetteer:read_gazetteer,lst:oaipmh:gazetteer:write_gazetteer}
dargestellt ist. % CHECK Reference

\begin{listing}
	\begin{minipage}{\textwidth}
		\begin{tcolorbox}[tlistingstyle]
			\inputminted[
				firstline=7,
				lastline=9,
				fontsize=\footnotesize,
			]{python}{../src/oaipmh/gazetteer.py}
		\end{tcolorbox}
		\subcaption[a]{\mintinline{python}{read_gazetteers}}% HACK
		\label{lst:oaipmh:gazetteer:read_gazetteers}
	\end{minipage}
	\begin{minipage}{\textwidth}
		\begin{tcolorbox}[tlistingstyle]
			\inputminted[
				firstline=28,
				lastline=28,
				fontsize=\footnotesize,
			]{python}{../src/oaipmh/gazetteer.py}
		\end{tcolorbox}
		\subcaption[b]{\mintinline{python}{read_gazetteer}}% HACK
		\label{lst:oaipmh:gazetteer:read_gazetteer}
	\end{minipage}
	\begin{minipage}{\textwidth}
		\begin{tcolorbox}[tlistingstyle]
			\inputminted[
				firstline=47,
				lastline=47,
				fontsize=\footnotesize,
			]{python}{../src/oaipmh/gazetteer.py}
		\end{tcolorbox}
		\subcaption[c]{\mintinline{python}{write_gazetteer}}% HACK
		\label{lst:oaipmh:gazetteer:write_gazetteer}
	\end{minipage}
	\nopagebreak
	\caption{Signaturen der Helferfunktionen in \mintinline{text}{src/oaipmh/gazetteer.py} für den Umgang mit \glspl{gazetteer}}
	\label{lst:oaipmh:gazetteer}
\end{listing}

\subsubsection{Extraktion der Metadaten}\label{ssec:dataharvesting}
Statt die Metadaten aus dem \gls{fid:linguistik}[-Portal]
zu extrahieren,
werden die Daten mit Hilfe von
\gls{oaipmharvest}
manuell
direkt von den Anbieterseiten extrahiert.
Die so gewonnenen \gls{xml}[-Dateien]
werden im nächsten Schritt weiter bearbeitet.
Durch diesen Weg kann auf die Besonderheiten jedes Anbieters eingegangen werden.


\begin{longlisting}
	\begin{tcolorbox}[tlistingstyle,breakable]
		\inputminted[
			firstline=22,
			lastline=41,
			fontsize=\footnotesize,
			escapeinside=||,
			gobble=10
		]{xml}{listings/oaixml:shortened:2022-07-13__oai_dc__000000000000.xml}
	\end{tcolorbox}%
	\nopagebreak
	\caption{Beispiel Metadaten eines \gls{oai-pmh}[-Eintrags] von \texttt{ubffm} innerhalb des \mintinline{xml}{oai_cd:dc}-Tags}
	\label{lst:oaipmh:xml:record:metadata}
\end{longlisting}

Da das \gls{oai-pmh}[-Protokoll] standardisiert ist,
treten Fallunterscheidungen erst auf,
wenn die eigentlichen Volltexte geladen werden.

Des Weiteren nutzt die \gls{jcs}
auch die \gls{url} der \gls{pdf}[-Dateien]
als \texttt{dc:identifier},
was dazu führt,
dass die Extraktion sehr einfach ist.
Andere Verlage haben die Links zu den eigentlichen \glspl{pdf}
nicht oder zumindest nicht klar erkennbar
in die \texttt{oai:metadata} Einträge ihrer
\texttt{oai:record} codiert.

\subsection{Erstellung einer \gls{huggingface:datasets}[-Klasse]}

Nachdem im vorherigen Schritt die Daten gewonnen wurden,
werden sie nun vorverarbeitet.
Hierfür werden Helferfunktionen erstellt
und in eine \gls{huggingface:datasets}[-Klasse] integiert.

\subsubsection{Extraktion der Daten aus den \gls{oai-pmh} \gls{xml}[-Dateien]}

Obwohl der Standard die Beschreibung der Elemente
durch das \gls{dublin-core}\footnote{\url{https://www.dublincore.org/specifications/dublin-core/dces/1999-07-02/}} vorschreibt,
gibt es z.B.\, bei \texttt{dc:format}
nur eine Empfehlung
für die Art des Inhalts.
Die \gls{jcs} folgt der Empfehlung und liefert den \gls{mime}[-Typ]
der Dateien,
während Language Science Press dies nicht tut.

Da bei \texttt{lang-sci-press} und per Definition nicht zwingend eine \gls{url} als \mintinline{xml}{dc:identifiers}
hinterlegt sein muss,
muss auch hier nachgefiltert werden.

Die Extraktion ist sowohl für \texttt{ubffm} als auch für \texttt{lang-sci-press}
als Quellen implementatiert.
Um nicht für alle Einträge des \glsxtrshort{html}[-Quellcodes] der \texttt{lang-sci-press} Webseite durchsuchen zu müssen,
werden auch die verwandten Resourcen (\mintinline{xml}{dc:relation}) extrahiert,
da hier bei \texttt{lang-sci-press} teilweise die \gls{pdf}[-Ressource] enthalten ist
(siehe \Cref{lst:ubffm:record}).

\begin{listing}
	\begin{tcolorbox}[tlistingstyle]
		\inputminted[
			fontsize=\footnotesize,
		]{xml}{listings/ubffm:record.xml}
	\end{tcolorbox}
	\caption{Beispielhafter \mintinline{xml}{dc:record} aus einer \gls{oai-pmh}[-Datei] von \texttt{ubffm}}
	\label{lst:ubffm:record}
\end{listing}

Ein Auswahl der Einträge aufgrund des Dateiformats ist bei den \texttt{ubffm} Einträgen möglich,
da als \mintinline{xml}{dc:format} entsprechend der Empfehlung der \gls{mime}[-Typ] der Ressource angegeben wird.
Da \texttt{lang-sci-press} das Format von Einträgen,
wie in \Cref{lst:lang-sci-press:record} zu sehen ist,
nicht als \gls{mime}[-Typ] angibt,
ist dieser Filter nicht implementiert.

\begin{listing}
	\begin{tcolorbox}[tlistingstyle]
		\inputminted[
			fontsize=\footnotesize,
			escapeinside=||,
		]{xml}{listings/lang-sci-press:record.xml}
	\end{tcolorbox}
	\caption{Beispielhafter gekürzter \mintinline{xml}{dc:record} aus einer \gls{oai-pmh}[-Datei] von \texttt{lang-sci-press}}
	\label{lst:lang-sci-press:record}
\end{listing}

\FloatBarrier
Das initiale Auslesen geschieht mittels der \gls{lxml}
Bibliothek, wie in \Cref{lst:xml_loader:load_xml_file} zu sehen ist.

\begin{longlisting}
	\begin{tcolorbox}[tlistingstyle,breakable]
		\inputminted[
			firstline=121,
			lastline=152,
			fontsize=\footnotesize
		]{python}{../src/oaipmh/xml_loader.py}
	\end{tcolorbox}
	\caption{Code zum Laden einer \gls{oai-pmh} \gls{xml}[-Datei]. \texttt{src/oaipmh/xml\_loader.py}}
	\label{lst:xml_loader:load_xml_file}
\end{longlisting}

Hierbei werden die eigentlichen Einträge jeweils mit der in \Cref{lst:xml_loader:_parse_record}
in das in \Cref{lst:xml_loader:OAIXMLRecordDict} dargestellte \texttt{OAIXMLRecordDict} Format umgewandelt.

\begin{longlisting} % CHECK longlisting?
\begin{tcolorbox}[tlistingstyle,breakable]
	\inputminted[
		firstline=93,
		lastline=118,
		fontsize=\footnotesize
	]{python}{../src/oaipmh/xml_loader.py}
\end{tcolorbox}
\caption{Code zum Umwandeln eines in \Cref{lst:xml_loader:load_xml_file} gewonnen \gls{oai-pmh}[-Record]
	in das in \Cref{lst:xml_loader:OAIXMLRecordDict} definierte Format. \texttt{src/oaipmh/xml\_loader.py}}
\label{lst:xml_loader:_parse_record}
\end{longlisting}


\begin{longlisting}% CHECK longlisting
	\begin{tcolorbox}[tlistingstyle,breakable]
		\inputminted[
			firstline=69,
			lastline=90,
			fontsize=\footnotesize
		]{python}{../src/oaipmh/xml_loader.py}
	\end{tcolorbox}
	\caption{Dedizierte Datenklasse für die einheitliche Typisierung der \gls{oai-pmh}[-Records]. \texttt{src/oaipmh/xml\_loader.py}}
	\label{lst:xml_loader:OAIXMLRecordDict}
\end{longlisting}

Die verschiedenen \texttt{get\_...} Funktionen,
die in \Cref{lst:xml_loader:load_xml_file,lst:xml_loader:_parse_record,lst:xml_loader:OAIXMLRecordDict}
verwendet werden,
sind \texttt{XPath} ausdrücke,
welche eine direkte Suche im \gls{xml}[-Baum] darstellen.
Mehr zur Syntax dieser Ausdrücke
kann \href{https://lxml.de/xpathxslt.html}{in der Dokumentation}\footnote{\url{https://lxml.de/xpathxslt.html}}
von \gls{lxml} gefunden werden.

In einem nächsten Schritt kann hier auch eine Extraktion des Volltextes geschehen.
Dafür muss je nach \gls{openarchivesinitiative}[-Anbieter]
der Link zur \gls{pdf}[-Datei] noch mittes \gls{beautiful-soup}
aus den in den \mintinline{xml}{dc:identifier}
referenzierten Webseiten extrahiert werden.
Der Code dafür ist in \Cref{lst:publishers:find_pdf_urls} dargestellt.

\begin{longlisting}
	\begin{tcolorbox}[tlistingstyle,breakable]
		\inputminted[
			firstline=14,
			lastline=67,
			fontsize=\footnotesize
		]{python}{../src/oaipmh/publishers.py}
	\end{tcolorbox}
	\caption{Code zum Finden der \gls{url}, welche auf \glspl{pdf} verweisen, in Abhängigkeit des Anbieters. \texttt{src/oaipmh/publishers.py}}
	\label{lst:publishers:find_pdf_urls}
\end{longlisting}

Da nicht alle Referenzen valide Weblinks sind,
wird mit \Cref{lst:_peak_url} eine Wrapper definiert,
der \mintinline{python}{None}
für diese zurückgibt.

\begin{longlisting}
	\begin{tcolorbox}[tlistingstyle,breakable]
		\inputminted[
			firstline=12,
			lastline=29,
			fontsize=\footnotesize
		]{python}{../src/oaipmh/helpers.py}
	\end{tcolorbox}
	\caption{Wrapper, um \gls{requests} in List Comprehensions mit Listen invalider Links zu nutzen \texttt{src/oaipmh/helpers.py}}
	\label{lst:_peak_url}
\end{longlisting}

Da bei den in \Cref{tbl:find_pdf_urls:failed} aufgeführten sieben der 1027 Einträgen
des \texttt{ubffm}-Providers
keine \gls{url} gefunden wird,
die auf eine entsprechende \gls{pdf} Datei verweist,
ist Fehlerbehandlung wichtig.


\subsubsection{Extraktion der Volltexte}
Da die Volltexte als \gls{pdf} vorliegen,
müssen die eigentlichen Texte extrahiert werden.
Hierfür gibt es verschiedenen Werkzeuge und Bibliotheken,
wie z.B.\, \gls{pypdf2}, \gls{pdfminer.six} oder \gls{pdftotext}.

Der erste Versuch der Extraktion wird mit \gls{pypdf2} durchgeführt.
Dieses analysiert die \gls{pdf} seitenweise.
Wenn dies nicht funktioniert,
wird ein erneuter Versuch mit \gls{pdfminer.six} gestartet.
Sollte dies auch fehlschlagen,
wird ein leerer Text zurückgegeben.
Die Implementierung ist in \Cref{lst:extract_pdf}
dargestellt.

\begin{longlisting}
	\begin{tcolorbox}[tlistingstyle,breakable]
		\inputminted[
			fontsize=\footnotesize
		]{python}{../src/oaipmh/extract_pdf.py}
	\end{tcolorbox}
	\caption{Code zur Extraktion des Textes aus einer \glspt{pdf}-Datei. \texttt{src/oaipmh/extract\_pdf.py}}
	\label{lst:extract_pdf}
\end{longlisting}

\subsubsection{Implementierung des Stringmatching}
\label{ssec:implementation:stringmatching}
In \Cref{lst:stringmatching} ist die Implementierung des Algorithmus,
der am Ende von \Cref{ssec:stringmatching} beschrieben wird,
zu sehen.

\begin{longlisting}
	\begin{tcolorbox}[tlistingstyle,breakable]
		\inputminted[
			fontsize=\footnotesize,
			firstline=2,
			lastline=107,
		]{python}{../src/oaipmh/dict_matcher.py}
	\end{tcolorbox}
	\caption[Implementierung des Stringmatching]{Implementierung des Stringmatching basierend auf \Cref{alg:stringmatching:optimized}. \texttt{src/oaipmh/dict\_matcher.py}}
	\label{lst:stringmatching}
\end{longlisting}

\subsubsection{Einbettung in \gls{huggingface:datasets}}
Um die Verwendung von \gls{huggingface:transformers}
zu vereinfachen,
werden die bisher definierten Methoden in \gls{huggingface:datasets} eingebunden.

Hierfür wird zunächst die \texttt{OAIPMHConfig} Konfigurationsklasse erstellt,
deren Parameter in \Cref{lst:oaipmhconfig} dargestellt sind.

\begin{longlisting}
	\begin{tcolorbox}[tlistingstyle,breakable]
		\inputminted[
			fontsize=\footnotesize,
			firstline=51,
			lastline=90,
		]{python}{../src/oaipmh/oaipmh.py}
	\end{tcolorbox}
	\caption[Beginn der Konfigurationsklasse \texttt{OAIPMHConfig}]{Beginn der Konfigurationsklasse \texttt{OAIPMHConfig} für das \texttt{OAIPMH} Dataset. \texttt{src/oaipmh/oaipmh.py}}
	\label{lst:oaipmhconfig}
\end{longlisting}


Die Fehlerbehandlung in \Cref{lst:extract_pdf},
die immer mindestens den leeren Text zurückgibt,
der am Anfang der Funktion definiert wird,
ist an dieser Stelle wichtig,
da die Funktion aus \Cref{lst:publishers:find_pdf_urls}
bei 8 der 1027 Einträgen auf Deutsch der \texttt{ubffm}
keine \gls{url} gefunden wird.
Die betreffenden \mintinline{xml}{dc:identifier} sind in \Cref{tbl:find_pdf_urls:failed}
aufgelistet.

% FIXME check entries
\begin{table}
	\begin{tabularx}{\textwidth}{l p{0.4\linewidth}}
		\mintinline{xml}{dc:identifier}                               & Besonderheit                                                                                          \\
		\mintinline{xml}{oai:publikationen.ub.uni-frankfurt.de:7101}  & Besteht aus einem Hauptband und einem Materialien Teil, die auf der Webseite referenziert werden \\
		\mintinline{xml}{oai:publikationen.ub.uni-frankfurt.de:47810} & Ist eine Zeitschrift                                                                             \\
		\mintinline{xml}{oai:publikationen.ub.uni-frankfurt.de:23611} & Ist eine Zeitschrift                                                                             \\
		\mintinline{xml}{oai:publikationen.ub.uni-frankfurt.de:4620}  & Ist eine Zeitschrift                                                                             \\
		\mintinline{xml}{oai:publikationen.ub.uni-frankfurt.de:30655} & Ist eine Zeitschrift                                                                             \\
		\mintinline{xml}{oai:publikationen.ub.uni-frankfurt.de:48364} & Ist eine Zeitschrift                                                                             \\
		\mintinline{xml}{oai:publikationen.ub.uni-frankfurt.de:33792} & Ist eine Zeitschrift                                                                             \\
		\mintinline{xml}{oai:publikationen.ub.uni-frankfurt.de:34780} & Besteht aus mehreren Dateien, die auf der Webseite referenziert werden                           \\
	\end{tabularx}
	\caption{\gls{oai-pmh}[-Records] der \texttt{ubffm},
		bei denen die Funktion aus \Cref{lst:publishers:find_pdf_urls} keinen Treffer liefert,
		und die zusätzliche Information über die Besonderheiten des Record
	}
	\label{tbl:find_pdf_urls:failed}
\end{table}



Zur Implementierung der \mintinline{python}{OAIPMH} Klasse
muss zum einen der Typ der Konfigurationsklasse definiert werden
und zum anderen müssen die \mintinline{python}{_info},
die \mintinline{python}{_split_generators}
und die \mintinline{python}{_generate_examples}
Methoden überschrieben werden.
In der \mintinline{python}{_info} Methode werden hierbei die Features festgelegt,
die das Dataset zurückgibt.
Die \mintinline{python}{_split_generators} Methode
liest die \gls{xml}[-Dateien]
mit der in \Cref{lst:xml_loader:load_xml_file} dargestellten Methhode
ein
und lädt potentiell die \glspl{pdf} herunter.
Die Volltextextraktion und das Stringmatching
werden potentiell in der \mintinline{python}{_generate_examples} Methode durchgeführt,
bevor die einzelnen Datensätze zurückgegeben werden.

Diese können danach durch die \gls{huggingface:transformers}[-Bibliothek] verwendet werden.


\subsection{Verwendung von \gls{huggingface:transformers}}
\subsubsection{Vorbereitung der Eingaben}

Zuerst wird der Text in Tokens aufgeteilt,
welche in numerische Werte übersetzt werden können.
Diese numerischen Werte können durch das neuronale Netzwerk verarbeitet werden,
da dieses aber eine Eingabe fixer Länge erwartet,
müssen zu lange Sequenzen abgeschnitten (\enquote{truncate})
und zu kurze Sequenzen ergänzt (\enquote{pad}) werden.
Die so bearbeiteten Werte werden anschießend in einen \gls{tensor} umgewandelt,
mit dem die weiteren Berechnungen durchgeführt werden können.
Diese werden wird von einem \enquote{preprocessor}
bzw.\, \enquote{tokenizer} ausgeführt.
\autocite{huggingface:docs:Transformers:preprocessing}

\subsubsection{Training}

Da potentiell die Grenzen der so erzeugten Token
nicht mit den Grenzen der Token des \gls{huggingface:datasets}
übereinstimmen,
müssen die Annotationen angepasst werden.
Die \mintinline{python}{tokenize_and_align_labels} Funktion,
die in \Cref{lst:train} dargestellt ist,
iteriert über die Token
und weisst dem jeweils ersten Token eines Worts die Annotation des Worts zu.\autocite{huggingface:course:chapter7:2}


\begin{longlisting}
	\begin{tcolorbox}[tlistingstyle,breakable]
		\inputminted[
			fontsize=\footnotesize
		]{python}{../src/train.py}
	\end{tcolorbox}
	\caption{Code zum Trainieren eines Modells \texttt{src/train.py}}
	\label{lst:train}
\end{longlisting}
