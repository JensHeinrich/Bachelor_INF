\documentclass[
german,
]{bachelor}

% Glossary support
\usepackage[
   record={nameref},
   nomain,
   style=tree,
]{glossaries-extra}

\usepackage{gls-repo}
\usepackage{gls-role}
\usepackage{gls-linklist}

% glossary formatting
\setabbreviationstyle[abbreviation]{long-short}
\setabbreviationstyle[acronym]{short-long}
%\setabbreviationstyle[licenses]{short-long}
\setabbreviationstyle[links]{short}
%\GlsXtrEnableLinkCounting[part]{general}
\glssetcategoryattribute{software}{textformat}{texttt}
%\glssetcategoryattribute{role}{textformat}{texttt}
\glssetcategoryattribute{filename}{textformat}{texttt}
\glssetcategoryattribute{jinja2var}{textformat}{texttt}
\GlsDeclareNoHyperList{filename,jinja2var}% suppress links for these types

% display k8s states in typewriter font
\glssetcategoryattribute{k8sstate}{textformat}{texttt}
\GlsDeclareNoHyperList{k8sstate}% suppress links for these types

\newglossary*{clusters}{Glossar --- QuestioningClusters}

\GlsXtrLoadResources[
   src={%
      glossaries/QuestioningClusters.bib%
   },%
   sort={en-GB},% sort according to 'en-GB' locale
   field-aliases={%
      description-de=description%
   },% load german description
   entry-type-aliases={%
      software=index,%
      k8sstate=index,%
      filename=index,%
   },% needed for bib2gls to recognize entries
   replicate-fields={%
      description-de=description,%
      name-de=name,%
      long=name,%
      %url=long,
      %url=description,
      %name=short,
      url=user1,%
   }, % optionally copy the values
   category={same as original entry},
   type={clusters},
   selection={all},
   %selection={recorded and deps and see},
   match={%
      %entrytype=software,%
      %%entrytype=entry,%
   },%
]

\GlsXtrLoadResources[
   src={%
      glossaries/institutes.bib,%
      glossaries/shortcuts.bib%
   },%
   entry-type-aliases={%
     abbreviation=index,%
   },%
   replicate-fields={%
     short-de=name,%
   },%
   field-aliases={%
      description-de=description,%
      short-de=short,%
      long-de=long,%
   },% load german description
   type={ignored},
   selection={all},
]

% Glossary support
\usepackage[
   record={nameref},
   nomain,
   style=tree,
]{glossaries-extra}

\usepackage{gls-repo}
\usepackage{gls-role}
\usepackage{gls-linklist}

% glossary formatting
\setabbreviationstyle[abbreviation]{long-short}
\setabbreviationstyle[acronym]{short-long}
%\setabbreviationstyle[licenses]{short-long}
\setabbreviationstyle[links]{short}
%\GlsXtrEnableLinkCounting[part]{general}
\glssetcategoryattribute{software}{textformat}{texttt}
%\glssetcategoryattribute{role}{textformat}{texttt}
\glssetcategoryattribute{filename}{textformat}{texttt}
\glssetcategoryattribute{jinja2var}{textformat}{texttt}
\GlsDeclareNoHyperList{filename,jinja2var}% suppress links for these types

% display k8s states in typewriter font
\glssetcategoryattribute{k8sstate}{textformat}{texttt}
\GlsDeclareNoHyperList{k8sstate}% suppress links for these types

\newglossary*{clusters}{Glossar --- QuestioningClusters}

\GlsXtrLoadResources[
   src={%
      glossaries/QuestioningClusters.bib%
   },%
   sort={en-GB},% sort according to 'en-GB' locale
   field-aliases={%
      description-de=description%
   },% load german description
   entry-type-aliases={%
      software=index,%
      k8sstate=index,%
      filename=index,%
   },% needed for bib2gls to recognize entries
   replicate-fields={%
      description-de=description,%
      name-de=name,%
      long=name,%
      %url=long,
      %url=description,
      %name=short,
      url=user1,%
   }, % optionally copy the values
   category={same as original entry},
   type={clusters},
   selection={all},
   %selection={recorded and deps and see},
   match={%
      %entrytype=software,%
      %%entrytype=entry,%
   },%
]

\GlsXtrLoadResources[
   src={%
      glossaries/institutes.bib,%
      glossaries/shortcuts.bib%
   },%
   entry-type-aliases={%
     abbreviation=index,%
   },%
   replicate-fields={%
     short-de=name,%
   },%
   field-aliases={%
      description-de=description,%
      short-de=short,%
      long-de=long,%
   },% load german description
   type={ignored},
   selection={all},
]


\author{Jens Heinrich}
\title{%
Analyse der Möglichkeit
einer automatischen Anreicherung von sprachwissenschatflichen Texten
durch Named Entity Recognition
unter Verwendung von BERT
mit Semantik aus der BLL Onthologie%
}

\usepackage{amsthm}
\newtheorem{prob}{Problemstellung}

\nocite{stack:tex:argmin}
\DeclareMathOperator*{\argmin}{arg\,min}
\usepackage{csquotes}

\begin{document}

\maketitle

\tableofcontents
\cleardoublepage

\section{Einleitung}

\section{Literaturrecherche}

\section{Grundlagen und Definitionen/Theorie}

\gls{maschine-learning} ist eine Unterkategorie von artificial intelligence % TODO gls
Das Ziel ist dem Computer die Möglichkeit zu geben zu lernen, ohne explizit dafür programmiert zu sein \autocite{levity:howdomachineslearn} % TODO mark citation

\citeauthor{levity:howdomachineslearn} unterteilt es in drei grosse Kategorien: 
\begin{itemize}
	\item Supervised learning % TODO gls
	\item Unsupervised learning % TODO gls
	\item Reinforcement learning % TODO gls
\end{itemize}

Beim Supervised learning % TODO gls
sind die Hauptaufgaben nach \autocite{levity:howdomachineslearn} Regression % TODO
und Klassifizierung % TODO gls

Beim Unsupervised Training ist das Ziel

Klassifizierung % TODO gls
ist der Prozess ein System mit bekannten annotierten Daten zu \enquote{trainieren} 
und so eine Reproduktion der Annotation auf nicht annotierten Daten zu erreichen.

Eine mögliche Annotation ist es den Wörtern eines Satzes ihre grammatikalische Funktion
oder andere semantische (?) Informationen % TODO 
zuzuordnen.
Wenn die semantische Information ist,
dass die annotierte Entität eine \enquote{spezielle} Entität ist, 
welche einen Namen hat,
dann spricht man von \gls{namedentityrecognition}.

Eine Art eine Liste solcher Entitäten zu erstellen,
ist es sie aus Wissensdatenbanken % TODO knowledge base glossary as knowledgebase
zu extrahieren.
In \autocite[A.1]{2006.15509} von \citetitle{2006.15509}
wird von \citeauthor{2006.15509} eine ganze Reihe von solchen knowledgebase % TODO gls
genannt.

Eine solche Liste von Entitäten wird oft als Gazzetteer bezeichnet % TODO gls
(siehe z.B. \autocite[Introduction]{Carlson2009}).



Der offensichtliche Weg eine Liste bekannter Entitäten über \enquote{string matching} zuzuordnen
hat die folgenden Probleme:
\begin{itemize}
	\item nur vorher bekannte Entitäten werden erkannt, 
	z.\,B. wird \enquote{Müller und Sohn} erkannt, 
	wenn es vorher aufgelistet wurde,
	aber der Transfer,
	dass auch \enquote{Meier und Sohn} eine Entität ist,
	kann nicht stattfinden
	\item Insbesondere in Grammatiken, 
	in denen Flexion % TODO glossary or wikipedia
	wichtig ist,
	kann die gleiche Entität mit verschiedenen Strings repräsentert werden.
	% TODO ref  ?
	\item die Laufzeit
	% TODO Laufzeit in Abhängigkeit von Dict und String ausdrücken
\end{itemize}

Eine Erweiterung dieses Ansatzes,
ist es die Liste zu erweitern,
wie es prinzipiell in \autocite[Abschnitt 3.4]{OASIcs-LDK-2019-11} 
für eine bessere Version der Worterkennung durchgeführt wird.
Hierbei wird die Erkennungsaufgabe von FST % TODO Glossary Finite State Transducern
erledigt,
welche eine Erkennung zuerst auf der Wortebene durchführen.
In unserem Beispiel von oben würden so die Worte 
\enquote{Müller}, \enquote{und} und \enquote{Sohn} auf der Zeichenebene erkannt werden,
und danach die Entität \enquote{Müller und Sohn} auf der Wortebene.

Wortebene ist hier tatsächlich schon nicht die vollständige Wahrheit,
denn in Sprachen,
die wie das Deutsche von zusammengesetzen Wörtern leben,
ist eine Einteilung in Wortbestandteile
(diese werden in \autocite{OASIcs-LDK-2019-11} als \foreignquote{english}{lemma} bezeichnet)
zielführender.

% TODO andere Möglichkeiten Darstellen

Der Ansatz mit der Wortebene zeigt auch schon auf einen konzeptuell-anderen Ansatz:
Es gibt bestimmte Formulierungen oder eher \enquote{Muster},
die oftmals eine Entität markieren.
So ist ein \enquote{\$NAME und \{Sohn,Söhne\}} insbesondere, 
wenn es in Anführungszeichen steht,
meistens eine Firma, also Kategorie ORG. % TODO NER Kategorien in gls

Auch Ketten von Grossbuchstaben sind oftmals Entitäten (Acronyme) % TODO gls




\subsection{Training}
Der Prozess des Trainings
% TODO
Es gibt verschiedene Möglichkeiten das Training durchzuführen




\subsection{\glspt{bllonthology}}
Was ist eine \gls{ontology}?
% TODO
Was ist die \gls{bll}?
% TODO

\subsection{Datenquellen}
% TODO Überlegen ob bllonthology hier hin passt
Was ist \gls{openaccess}?
Verwendung des \gls{oai-pmh} zum Sammeln von Informationen
mithilfe des \gls{oaipmharvest}
% TODO



\subsection{Datapreparation}
\subsubsection{Datacollection}
Eine Liste der potent
\subsubsection{Metadataextraction}
\subsubsection{Textextraction} (Optional)
\subsubsection{Stringmatching}



\subsection{Named Entity Recognition}
\subsubsection{Tokenization}
Die Eingabe für BERT % TODO gls
% TODO oder ein anderes Netzwerk (?)

\subsubsection{Tokenclassification}

\subsection{Named Entity Linking}
Der Prozess der \gls{namedentityrecognition}
lässt sich nach \autocite{2006.15509} als Lösung des folgenden mathematischen Problems beschreiben:

\begin{prob}{\glspt{namedentityrecognition}}\label{prob:nlp:ner}
	Für ein \gls{nlp:sentence} \glsentryformula{nlp:sentence}
	wird eine \gls{nlp:label_sequence} \glsentryformula{nlp:label_sequence} gesucht,
	sodass \(
	\glssymbol{nlp:label}_i \in
	\left\lbrace
	\glssymbol{nlp:label:beginning}\texttt{-X},
	\glssymbol{nlp:label:inside}\texttt{-X},
	\glssymbol{nlp:label:outside}
	:
	\text{\texttt{X} ist Entitätstyp}
	\right\rbrace
	\)
	und gilt:
	\[
		\glssymbol{nlp:label}_i =	
		\begin{cases}
			\texttt{\glssymbol{nlp:label:beginning}-X},
			\text{wenn \(\glssymbol{nlp:token}_i\)
				der Beginn einer Entität vom Typ \(\texttt{X}\) ist
			} \\
			\texttt{\glssymbol{nlp:label:inside}-X},
			\text{wenn \(\glssymbol{nlp:token}_i\)
				ein Token einer Entität vom Typ \(\texttt{X}\)
				und nicht deren Beginn ist
			} \\
			\glssymbol{nlp:label:outside}, \text{sonst}.
		\end{cases}
	\]
	Diese \texttt{BIO}-Klassifikation geht auf \citeauthor{10.1145/2396761.2398506} zurück \autocite{10.1145/2396761.2398506}.
\end{prob}

Seien \(M\) bereits annotierte \glslink{nlp:sentence}{Sätze}
\glsentryformula{nlp:labelled_sentences} gegeben
und beschreibe \glsentryformula{nlp:model} ein \gls{namedentityrecognition}[-Modell],
wobei \glssymbol{nlp:sentence} ein Satz
und \glssymbol{nlp:model:parameters} die Parameter des Modells sind.
Dann lässt sich anhand der Problemstellung~\cref{prob:nlp:ner}
die Güte des Modells über eine Metrik bestimmen.
Eine beliebte Metrik im Rahmen des \gls{machine-learning} ist die \gls{crossentropyloss}[-Funktion],
wie sie in \autocite[5.5]{juraksky2000speech} definiert wird:
\glsentryformula{crossentropyloss}.
Hierbei wäre der ideale Fall,
dass jeder Satz
\gls{nlp:label} entsprechend der vorgegeben (idealen) Annotation
zugewiesen bekommt.
Dementsprechend werden die \gls{nlp:model:parameters} so gesucht,
dass sie diese Distanz über alle Trainingsdaten minimieren.
Somit sind die idealen Parameter durch die Formel
\autocite[1]{2006.15509}
gegeben:
\begin{equation}
	\hat{\glssymbol{nlp:model:parameters}} : =
	\argmin_{\glssymbol{nlp:model:parameters}}
	\frac{1}{M}
	\sum_{m=1}^{M}
	\glssymbol{crossentropyloss}\left(
	\glssymbol{nlp:label_sequence}_m,
	\glssymbol{nlp:model}\left(
		\glssymbol{nlp:sentence}_m;
		\glssymbol{nlp:model:parameters}
		\right)
	\right)
\end{equation}

\subsection{fernüberwachte \glspt{namedentityrecognition}}
Hierbei sind im Gegensatz zur \enquote{klassischen} \gls{namedentityrecognition}
% TODO das klingt unrund
nicht die \gls{nlp:token} selber beschriftet,
sondern der gesamte Text wird automatisch mit \glspl{nlp:label} versehen.
\citeauthor{2006.15509} listet als Möglichkeiten hierfür
\enquote{string matching}, \enquote{regexp} und heuristische Verfahren
\autocite{2006.15509}.

\section{Vorherige Arbeiten}


\section{Methodik/Umsetzung}

\subsection{Erstellung eines Gazetteers aus der \glspt{bllonthology}} % TODO gls
Um eine initiale Annotation der Texte durchzuführen,
wurde aus der \gls{bllonthology} ein Gazetteer % TODO gls
der Anzeigenamen der Objekte erstellt.

\subsection{Vorbereitung der Beispieldokumente}
Da die Beispieldokumente als \gls{pdf} vorliegen,
müssen die eigentlichen Texte extrahiert werden.
Hierfür gibt es verschiedenen Werkzeuge und Bibliotheken,
wie z.B.\, \gls{pypdf2} oder \gls{pdftotext}.

Der erste Versuch der Extraktion wird mit \gls{pypdf2} durchgeführt.
Dieses analysiert die \gls{pdf} seitenweise.
Der kommentierte Code findet sich in \texttt{src/extract\_fulltexts.ipynb}.

\subsection{\glspt{naturallanguageprocessing} mit \glspt{BERT}}
Der Prozess des \gls{naturallanguageprocessing} mit \gls{BERT} besteht aus verschiedenen Schritten.
Zuerst wird der Text in Tokens aufgeteilt,
welche in numerische Werte übersetzt werden können.
Diese numerischen Werte können durch das neuronale Netzwerk verarbeitet werden,
da dieses aber eine Eingabe fixer Länge erwartet,
müssen zu lange Sequenzen abgeschnitten (\enquote{truncate})
und zu kurze Sequenzen ergänzt (\enquote{pad}) werden.
Die so bearbeiteten Werte werden dann in einen \gls{tensor} umgewandelt
mit dem die weiteren Berechnungen durchgeführt werden können.
Diese Schritte wird von einem \enquote{preprocessor}
bzw.\, \enquote{tokenizer} ausgeführt.
\autocite{huggingface:docs:Transformers:preprocessing}
Eine beliebte Bibliothek hierfür ist \gls{huggingface:tokenizers}.



Der darauf folgende Schritt ist die Auswertung durch
% TODO



\section{Diskussion}

\section{Fazit}

\appendix

\printunsrtglossaries
\end{document}
